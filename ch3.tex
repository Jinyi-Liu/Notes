\numberwithin{equation}{section}
\newcommand{\spq}{S(q,p^o(q))}
\newcommand{\pq}{p^o(q)}
\thispagestyle{empty}
\section{Coordinating the newsvendor with \textit{price-dependent} demand}


\subsection{Model and analysis}
Now the retailer chooses his price in addition to his order quantity. Let $F(q|p)$ be the distribution function of demand, where $p$ is the retail price. Assume $\frac{\partial F(q|p)}{\partial p}>0$. To obtain initial insights, assume the retailer sets his \textbf{price} \textit{at the same time as} his \textbf{stocking decision} and the price is \textbf{fixed} throughout the season.
\begin{note}
    van Mieghem and Dada (1999). A hybrid model. The retailer chooses $q$, then observes a demand signal and then chooses price.
\end{note}

The integrated channel's profit is 
\begin{equation*}
    \Pi(q,p)=(p-v+g)S(q,p)-(c-v)q-g\mu
\end{equation*}
where $S(q,p)$ is expected sales given the stocking quantity $q$ and the price $p$, and similarly, we have
\begin{equation*}
    S(q,p)=q-\int_0^q F(y|p)dy
\end{equation*}
\begin{note}
    The integrated channel profit function need not be concave nor unimodal (Petruzzi \& Dada 1999)
\end{note}
Let $\pq$ be the supply chain optimal price for a given $q$. The necessary condition for coordination is
\begin{equation}\label{eq:3.1}
    \frac{\partial\Pi(q,p^o(q))}{\partial p}=S(q,p^o(q))+(p^o(q)-v+g)\frac{\partial S(q,p^o(q))}{\partial p}=0.
\end{equation}
\begin{note}
    Either not satisfy the first-order condition or fail to coordinate the quantity decision.
\end{note}

Consider the \textbf{quantity-flexibility} contract. The retailer's profit function is 
\begin{align*}
    \pi_r(q,p,w_q,\delta)=&(p-v+g_r)S(q,p)-(w_q+c_r-v)q\\
    +&(w_q+c_r-v)\int_{(1-\delta)q}^q F(y|p)dy-\mu g_r
\end{align*}
For price coordination the first-order condition must hold,

\begin{align}
\frac{\partial \pi_{\mathrm{r}}\left(q, p^{\mathrm{o}}(q), w_{q}, \delta\right)}{\partial p}=& S\left(q, p^{\mathrm{o}}(q)\right)+\left(p^{\mathrm{o}}(q)-v+g_{\mathrm{r}}\right) \frac{\partial S\left(q, p^{\mathrm{o}}(q)\right)}{\partial p} \nonumber\\
&+\left(w_{q}+c_{\mathrm{r}}-v\right) \int_{(1-\delta) q}^{q} \frac{\partial F\left(y \mid p^{\mathrm{o}}(q)\right)}{\partial p} \mathrm{~d} y \nonumber\\
=& 0\label{eq:3.2}
\end{align}
The second term in \autoref{eq:3.2} is no smaller than the second term in \autoref{eq:3.1}\footnote{The assumption of $\partial F(q|p)/\partial p>0.$}, so the above holds only if the third term is nonpositive. 
But the third term is nonnegative as $w_q+c_r-v\geq 0$, so with a coordinating $w_q$, the coordination of price can only occur if $g_s=0$ and either $w_q=v-c_r$ or $\delta=0$. \textbf{Neither} is desirable. With $w_q=v-c_r$, then supplier has $w_q<c_s$\footnote{Why? An assumption?} which is not acceptable. With $\delta=0$ the contract degenerates to just a wholesale-price contract, so the retailer's quantity action is not optimal. Hence, the quantity-flexibility contract does not coordinate the newsvendor with price-dependent demand.

The \textbf{sales-rebate} contract does not fare better:
\begin{align*}
    \frac{\partial\pi_r(q,\pq,w_s,r,t)}{\partial p}=&\spq+(\pq-v+g_r)\frac{\partial\spq}{\partial p}\\
    &-r\int_t^q\frac{\partial F(y|\pq)}{\partial p}dy
\end{align*}
Since the last term is negative when $r>0$ and $t<q$, we know that the retailer prices below the optimal price\footnote{The above derivative is negative. Why it means that the retailer prices below the optimal price?}. Coordination might be achieved if there is something to induce the retailer to a higher price.

Now consider a \textbf{buyback} contract. The retailer's profit function is 
\begin{equation*}
    \pi_r(q,p,w_b,b)=(p-v+g_r-b)S(q,p)-(w_b-b+c_r-v)q-g_r\mu.
\end{equation*}
For coordination we must have the first-order condition:
\begin{equation}\label{eq:3.3}
    \frac{\partial\pi_r(q,\pq,w_b,r,t)}{\partial p}=\spq+(\pq-v+g_r-b)\frac{\partial\spq}{\partial p}=0.
\end{equation}
But comparing with \autoref{eq:3.1} it holds only if $b=-g_s< 0$ which violates that $b\geq 0$\footnote{If $g_s=0$, then $w_b=c_s$ and $b_s=0$ which means that the supplier earns no positive profit.}. Therefore, a buyback contract does not coordinate the newsvendor with price-dependent demand.

The buyback contract fails to coordinate in this setting because the parameters of the coordinating contracts depend on the price: from \autoref{eq:2.5} and \autoref{eq:2.6}, the coordinating parameters are
\begin{align*}
    b&=(1-\lambda)(p-v+g)-g_s\\
    w_b&=\lambda c_s+(1-\lambda)(p+g-c_r)-g_s.
\end{align*}
For a fixed $\lambda$, the coordianting buyback rate and wholesale price are linear in $p$. Hence, the buyback contract coordiantes the newsvendor with price-dependent demand if $b$ and $w_b$ are made \textbf{contingent} on the retail price chosen, or if $b$ and $w_b$ are chosen \textbf{after} the retailer commits to a price (but before the retailer chooses $q$). This is the \textbf{price-discount-sharing} contract\footnote{Bernstein and Federgruen (2000)}, which is called a "bill back" in practice. The retailer gets a lower wholesale price if the retailer reduces his price, i.e., the supplier shares in the cost of a price discount with the retailer. Then we have the retailer profit function:
\begin{align*}
    \pi_r(q,p,w_b,b)&=\lambda(p-v+g)S(q,p)-\lambda(c-v)q-g_r\mu\\
    &=\lambda(\Pi(q,p)+g\mu)-g_r\mu
\end{align*}
Hence, for the retailer as well ass the supplier, $\{q^o,p^o\}$ is optimal for $\lambda\in[0,1]$.

Now consider the \textbf{revenue-sharing} contract. The retailer's profit is 
\begin{equation*}
    \pi_r(q,p,w_r,\phi)=(\phi(p-v)+g_r)S(q,p)-(w_r+c_r-\phi v)q-g_r\mu.
\end{equation*}
Coordination require
\begin{equation}\label{eq:3.4}
    \frac{\pi_r(q,\pq,w_r,\phi)}{\partial p}=\spq+(\pq-v+g_r/\phi)\frac{\partial\spq}{\partial p}=0.
\end{equation}

\begin{itemize}
    \item Consider $g_r=g_s=0$. In this situation,
    \begin{equation*}
        \frac{\partial\pi_r(q,p,w_r,\phi)}{\partial p}=\frac{\partial\Pi(q,p)}{\partial p}
    \end{equation*}
    with \textbf{any} revenue-sharing contract. Thus, the retailer chooses $\pq$ no matter which revenue-sharing contract is chosen. Now revenue sharing is able to coordinate the retailer's quantity decision with precisely the same set of contracts used when the retailer prices is fixed. 

    Recall that with the \textit{fixed price} newsvendor \textbf{revenue sharing} and \textbf{buybacks} are equivalent. Here, the contracts produce different outcomes because with a buyback the retailer's share of revenue $(1-b/p)$ depends on the price, whereas with revenue sharing it is independent of the price, by definition\footnote{The above partial derivative}. However, the \textbf{price contingent buyback} contract (\textbf{price-discount} contract) is equivalent to revenue sharing: if $g_r=g_s=0$, the coordinating revenue-sharing contract yield 
    \begin{equation*}
        \pi_r(q,p,w_r,\phi)=\lambda\Pi(q,p)
    \end{equation*}
    from \autoref{eq:2.10}. And the price contingent buyback contract yield the same profit for any quantity and price from \autoref{eq:2.7},
    \begin{equation*}
        \pi_r(q,p,b(p),w_b(p))=\lambda\Pi(q,p).
    \end{equation*}
    \item Consider either $g_r>0$ or $g_s>0$.
\end{itemize}

























