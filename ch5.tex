\section{Coordination with multiple newsvendors}
This section considers two models with one supplier and multiple competing retailers. 

\subsection{Competing newsvendors with a fixed retail price}
Set $c_r=g_r=g_s=v=0$, increase the number of retailers to $n>1$. $D$ the total retail demand. And for each retailer $i$'s demand:
\begin{equation*}
    D_i=\left(\frac{q_i}{q}\right)D,
\end{equation*}
where $q=\sum_{i=1}^n q_i$ and $q_{-i}=q-q_i$. 
Given the proportional allocation rule, the integrated supply chain faces a single newsvendor problem. Hence we have 
\begin{equation}
    F(q^o)=\frac{p-c}{p}.
\end{equation}
Retailer $i$'s profit function with a buyback contract is 
\begin{equation*}
    \pi_i(q_i,q_{-i})=(p-w)q_i-(p-b)\left(\frac{q_i}{q}\right)\int_0^q F(x)dx.
\end{equation*}
The above also provides the retailer's profit with a wholesale-price contract (i.e., set $b=0$). It's strictly concave in $q$. Hence, for every $q_{-i}$ there is a unique optimal response. Consider a Nash equilibrium $\{q^*_i\}_{i=1}^n$, it must have
\begin{equation*}
    \frac{\partial\pi_i(q_i,q_{-i})}{\partial q_i}=q^*\left(\frac{p-w}{p-b}\right)-q^*_i F(q^*)-q_{-i}^*\left(\frac{1}{q^*}\int_0^{q^*}F(x)dx\right)=0.
\end{equation*}
Substitute $q_{-i}^{*}=q^{*}-q_{i}^{*}$ into the above equation and solve for $q_{i}^{*}$ given a fixed $q^{*}$ :
\begin{equation}\label{eq:5.2}
    q_{i}^{*}=q^{*} \frac{\left((p-w) /(p-b)-\left(1 / q^{*}\right) \int_{0}^{q^{*}} F(x) \mathrm{d} x\right)}{F\left(q^{*}\right)-1 / q^{*} \int_{0}^{q^{*}} F(x) \mathrm{d} x} .
\end{equation}
Now substitute it into $q^*=n q_i^*$, then we have
\begin{equation}\label{eq:5.3}
    g(q^*)\equiv\frac{1}{n}F(q^*)+\left(\frac{n-1}{n}\right)\left(\frac{1}{q^*}\int_0^{q^*}F(x)dx\right)=\frac{p-w}{p-b}.
\end{equation}
It's easy to see that $g(0)=1$, $g(\infty)=1$ and $g^\prime(\cdot)>0$. Thus, when $b<w<p$, there exists a unique $q^*$ satisfying \autoref{eq:5.3}.

Consider $n$. LHS in \autoref{eq:5.3} is decreasing in $n$, thus $q^*$ is increasing in $n$. Competition makes the retailers order more inventory because of the \textbf{demand-stealing effect}: each retailer \textbf{ignores} the fact that ordering more means the other retailers' demands \textbf{stochastically decrease}.

Due to the \textbf{demand-stealing effect} the supplier can coordinate the supply chain and earn a positive profit with just a wholesale-price contract.
Let $\hat{w}(q)$ be the wholesale price that induces the retailers to order $q$ units with a wholesale-price contract (i.e., with $b=0$). From \autoref{eq:5.3},
\begin{equation*}
    \hat{w}(q)=p\left(1-\left(\frac{1}{n}\right) F(q)-\left(\frac{n-1}{n}\right)\left(\frac{1}{q} \int_{0}^{q} F(x) \mathrm{d} x\right)\right).
\end{equation*}
By definition $\hat{w}(q^o)$ is the coordinating wholesale price. Given $F(q^o)=(p-c)/c$ and 
\begin{equation*}
    \frac{1}{q}\int_0^q F(x)dx<F(q),
\end{equation*}
it can be shown that $\hat{w}(q^o)>c$ when $n>1$\footnote{
    $\hat{w}(q^o)>p\left(1-F(q^o)\right)=p(1-\frac{p-c}{c})=\frac{2cp-p^2}{c}$ ?????????\textbf{TBD.}
}. Hence, the supplier earns a positive profit. But with the \textbf{single} retailer model channel coordination is only achieved when the supplier earns zero profit, i.e., $\hat{w}(q^o)=c$. 

But the coordination is not optimal for supplier. The profit function is
\begin{equation*}
    \pi_s(q,\hat{w}(q))=q(\hat{w}(q)-c).
\end{equation*}
Assuming $n>1$, we have 
\begin{equation*}
    \frac{\partial\pi_s(q^o,\hat{w}(q^o))}{\partial q}=-\frac{q^o p f(q^o)}{n}<0.
\end{equation*}







