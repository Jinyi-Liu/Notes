\newpage
\section{Coordinating the newsvendor with demand updating}

\subsection{Model and analysis}
\begin{note}
    Donohue 2020.
\end{note}
Let $\xi\geq 0$ be the realization of that demand signal. Let $G(\cdot)$ be its distribution function and $g(\cdot)$ its density function. Let $F(\cdot\mid\xi)$ be the distribution given signal and it's stochastically increasing in $\xi$. Let period $1$ be the time before the demand signal and $2$ the time between the demand signal and the start of the selling season. 

Let $q_i$ be the retailer's total order as of period $i$\footnote{$q_1$ and $q_2-q_1$}. Let $c_i$ be the supplier's per unit production cost in period $i$, with $c_1<c_2$. The supplier charges $w_i$ in period $i$. Also, the supplier offers buyback for $b$ per unit unsold. Let $p$ be the retail price. Normalize to zero the salvage value of leftover inventory and any indirect costs due to lost sales. No holding cost on inventory carried from period 1 to period 2.

Begin with period $2$. Let $\Omega_2(q_2\mid q_1,\xi)$ be the supply chain's expected revenue minus the period 2 production cost:
\begin{equation}
    \Omega(q_2\mid q_1,\xi)=p S(q_2\mid \xi)-c_2 q_2+c_2 q_1.
\end{equation}
Let $q_2(q_1,\xi)$ be the supply chain's optimal $q_2$ given $q_1$ and $\xi$. Let $q_2(\xi)=q_2(0,\xi)$, i.e., $q_2(\xi)$ is the optimal order if the retailer has no inventory at the start of period 2. Given $\Omega_2(q_2\mid q_1,\xi)$ is strictly concave in $q_2$,
\begin{equation}
    F(q_2(\xi)\mid\xi)=\frac{p-c_2}{p}.
\end{equation}
$q_2(\xi)$ is increasing in $\xi$\footnote{$F(\cdot\mid \xi)$ is SI in $\xi$ and RHS is unchanged in $\xi$.}, so it is possible to define the function $\xi(q_1)$ such that 
\begin{equation}
    F(q_1\mid\xi(q_1))=\frac{p-c_2}{p}.
\end{equation}

We have the retailer's period 2 expected profit
\begin{equation*}
    \pi_2(q_2\mid q_1,\xi)=(p-b)S(q_2\mid \xi)-(w_2-b)q_2+w_2 q_1,
\end{equation*}
where assume the supplier delivers the retailer's order in full. Choose $\lambda\in[0,1]$ and 
\begin{align*}
    p-b&=\lambda p\\
    w_2-b&=\lambda c_2.
\end{align*}
With this contract we have 
\begin{equation*}
    \pi_2(q_2\mid q_1,\xi)=\lambda(\Omega_2(q_2\mid q_1,\xi)-c_2 q_1)+w_2 q_1.
\end{equation*}
Thus, $q_2(q_1,\xi)$ is also the retailer's optimal order, i.e., the contract coordinates the retailer's \textbf{period 2 decision.}

Now consider whether the \textbf{supplier} indeed fills the retailer's entire period 2 order. Let $x$ be the total inventory in the supply chain at the start of period 2 with $x\geq q_1$. Let $y$ be the inventory at the retailer after the supplier's delivery in period 2. Let $\Pi_2(y\mid x, q_1,\xi)$ be the supplier's profit, where $x\leq y\leq q_2$,
\begin{align*}
    \Pi_2(y\mid x,q_1,q_2,\xi)&=b S(y\mid\xi)-by+w_2(y-q_1)-(y-x)c_2\\
    &=(1-\lambda)(\Omega_2(y\mid q_1,\xi)-c_2 q_1)+c_2 x- w_2 q_1
\end{align*}
where the above follows from the contract terms, $w_2=\lambda c_2+b$. Given $q_2>x$, the supplier fills the order as long as $q_2\leq q_2(q_1,\xi)$. 

In \textbf{period 2}, assuming a coordinating $\{w_2,b\}$ pair is chosen, the retailer's expected profit is\footnote{$-w_1 q_1+E\left[\pi_2(q_2\mid q_1,\xi)\right]$} 
\begin{equation*}
    \pi_1(q_1)=-(w_1-w_2+\lambda c_2)q_1+\lambda E\left[\Omega_2(q_2(q_1,\xi)\mid q_1,\xi)\right].
\end{equation*}
The supply chain's expected profit is 
\begin{equation*}
    \Omega_1(q_1)=-c_1 q_1+E\left[\Omega_2(q_2(q_1,\xi)\mid q_1,\xi)\right].
\end{equation*}
Choose $w_1$ so that 
\begin{equation*}
    w_1-w_2+\lambda c_2=\lambda c_1
\end{equation*}
because then
\begin{equation*}
    \pi_1(q_1)=\lambda \Omega_1(q_1),
\end{equation*}
i.e., the supply chain coordinates.
Given $\Omega_1(q_1)$ is strictly concave, $q_1^o$ follows:
\begin{align}
     \frac{\partial\Omega_1(q_1^o)}{\partial q_1}&=-c_1+c_2(1-G(\xi(q_1^o)))+\int_0^{\xi(q_1^o)} p S^\prime(q_1^o\mid\xi) g(\xi)d \xi \nonumber\\
    &=0 \label{eq:6.4}
\end{align}


Assuming the supplier fills the retailer's period 2 order, the supplier's period 2 profit is 
$$
\begin{aligned}
\Pi_{2}\left(x, q_{1}, q_{2}, \xi\right) &=b S\left(q_{2} \mid \xi\right)-b q_{2}-\left(q_{2}-x\right)^{+} c_{2} \\
&=(1-\lambda) \Omega_{2}\left(q_{2} \mid q_{1}, \xi\right)-w_{2} q_{2}+x c_{2}-\left(x-q_{2}\right)^{+} c_{2}
\end{aligned}
$$
Given that $q_{2} \geq q_{1}$, the above is strictly increasing in $x$ for $x \leq q_{1}$. Hence, the supplier surely produces and delivers the retailer's period 1 order (as long as $\left.q_{1} \leq q_{1}^{o}\right)$. The supplier's period 1 expected profit is
$$
\begin{aligned}
\Pi_{1}\left(x \mid q_{1}\right)=&-c_{1} x+E\left[\Pi_{2}\left(x, q_{1}, q_{2}, \xi\right)\right] \\
=&-c_{1} x+E\left[(1-\lambda) \Omega_{2}\left(q_{2} \mid q_{1}, \xi\right)\right]-w_{2} q_{2}+x c_{2} \\
&-c_{2} \int_{0}^{\xi(x)}\left(x-q_{2}(\xi)\right) g(\xi) \mathrm{d} \xi
\end{aligned}
$$
It follows that
$$
\frac{\partial \Pi_{1}\left(x \mid q_{1}\right)}{\partial x}=-c_{1}+c_{2}(1-G(\xi(x)))
$$
and from \autoref{eq:6.4}
$$
\frac{\partial \Pi_{1}\left(q_{1}^{\mathrm{o}} \mid q_{1}^{\mathrm{o}}\right)}{\partial x}=-c_{1}+c_{2}\left(1-G\left(\xi\left(q_{1}^{\mathrm{o}}\right)\right)\right)<0 ,
$$
i.e., the supplier has no incentive to produce more than $q_1^o$ given the retailer orders $q_1^o$. Hence, with a coordinating $\{w_1,w_2,b\}$ contract the supplier produces just enough inventory to cover the retailer's period 1 order. 
\begin{note}
    $$w_2-c_2=w_1-(\lambda c_1+(1-\lambda)c_2)<w_1-c_1,$$
    i.e. with a coordinating contract the supplier's margin in period 2 is actually \textbf{lower} than in period 1, which contrasts with intuition that the supplier should charge a higher margin for the later production since it offers the retailer an additional benefit over early production.
\end{note}



