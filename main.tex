\documentclass[12pt]{article}
\usepackage[utf8]{inputenc}	
\usepackage{amsmath,amsthm,amsfonts,amssymb,amscd}
\usepackage{multirow,booktabs}
\usepackage[table]{xcolor}
\usepackage{fullpage}
\usepackage{lastpage}
\usepackage{enumitem}
\usepackage{fancyhdr}
\usepackage{mathrsfs}
\usepackage{wrapfig}
\usepackage{setspace}
\usepackage{calc}
\usepackage{multicol}
\usepackage{cancel}
\usepackage[retainorgcmds]{IEEEtrantools}
\usepackage[margin=4cm]{geometry}
\usepackage{amsmath}
\newlength{\tabcont}
\setlength{\parindent}{0.0in}
\setlength{\parskip}{0.00in}
\usepackage{empheq}
\usepackage{framed}
\usepackage[most]{tcolorbox}
\usepackage{xcolor}
\usepackage{hyperref}
\usepackage{palatino}

% \usepackage[style=alphabetic,sorting=nyt,sortcites=true,autopunct=true,babel=hyphen,hyperref=true,abbreviate=false,backref=true,backend=biber]{biblatex}
% \addbibresource{bibliography.bib}

\colorlet{shadecolor}{orange!15}
\parindent 0in
\parskip 8pt
% \geometry{margin=1in, headsep=0.25in}
\theoremstyle{definition}
\newtheorem{defn}{Definition}
\newtheorem{reg}{Rule}
\newtheorem{exer}{Exercise}
\newtheorem{note}{Note}

% \usepackage{amsmath}
\DeclareMathOperator*{\argmax}{arg\,max}
\DeclareMathOperator*{\argmin}{arg\,min}
\newcommand{\oF}{\overline{F}(q)}

\begin{document}



\setcounter{section}{1}
\numberwithin{equation}{section}

% \title{Chapter 2 Review Notes}
% \maketitle
\thispagestyle{empty}

\begin{center}
{\LARGE \bf Notes}\\
% {\large }\\
\end{center}

\section{Coordinating the newsvendor}

With the standard wholesale-price contract, it is shown that the retailer does not order enough inventory to maximize the supply chain's total profit because the retailer ignores the impact of his action on the supplier's profit. Hence, coordination requires that the retailer be given an **incentive** to increase his order

Types of contracts to coordinate the supply chain and arbitrarily divide its profit:
- Buyback contracts
- revenue-sharing contracts
- quantity-flexibility contracts 
- sales-rebate contracts
- quantity-discount contracts


\subsection{Model and analysis}
$\mu=E[D]$ is the mean of demand. The supplier's production cost per unit is $c_s$ and the retailer's marginal cost per unit is $c_r$, $c_s+c_r<p$. $c_r$ is incurred upon procuring a unit. Goodwill penalty cost $g_r$ and the analogous cost for the supplier is $g_s$. Let $c=c_s+c_r$ and $g=g_s+g_r$. $v$ is net of any salvage expenses. 

The details of the negotiation process is not explored.

Each firm is risk neutral. Full information.

- voluntary compliance
- forced compliance
- The approach taken in this section is to assume forced compliance but to check if the supplier has an incentive to deviate from the proposed contractual terms.


\begin{align*}
    S(q)&=E[\min(q,D)]=q(1-F(q))+\int_0^q y f(y)dy=q-\int_0^q F(y)dy\\
    I(q)&=E[(q-D)^+]=q-S(q)\\
    L(q)&=E[(D-q)^+]=\mu-S(q)
\end{align*}

where $I(q)$ is the expected leftover inventory and $L(q)$ is the lost-sales function.

The retailer's profit function is 
\begin{equation}\label{eq:retailer}\tag{Retailer}
    \begin{aligned}
        \pi_r(q)&=pS(q)+v I(q)-g_r L(q)-c_r q - T\\
    &=(p-v+g_r)S(q)-(c_r-v)q-g_r\mu-T,
    \end{aligned}
\end{equation}
the supplier's profit function is 
\begin{equation}\label{eq:supplier}\tag{Supplier}
    \pi_s(q)=g_s S(q)-c_s q-g_s\mu + T,
\end{equation}
and the supply chain's profit function is 
\begin{equation}\label{eq:2.1}
    \Pi(q)=\pi_r(q)+\pi_s(q)=(p-v+g)S(q)-(c-v)q-g\mu
\end{equation}

Let $q^o$ be a supply chain optimal order quantity, we have 
\begin{equation}\label{eq:2.2}
    S^\prime(q^o)=\overline F(q^o)=\frac{c-v}{p-v+g}
\end{equation}

since $F$ is strictly increasing and thus $\Pi$ is strictly concave and the optimal order quantity is unique.

Let $q_r^*=\argmax \pi_r(q)$

\subsection{The wholesale-price contract}\
Let $T_w(q,w)=w q$. Since $\pi_r(q,w)$ is strictly concave in $q$, we have 

\begin{equation}\label{eq:2.3}
    (p-v+g_r)S^\prime(q_r^*)-(w+c_r-v)=0.
\end{equation}

Since $S^\prime(q)$ is decreasing, $q_r^*=q^o$ only when
$$
w=(\frac{p-v+g_r}{p-v+g})(c-v)-(c_r-v).
$$
It shows that $w\leq c_s$, i.e., coordinates only if the supplier earns a \textbf{nonpositive} profit. Thus the wholesale-price contract is generally \textbf{not considered} a coordinating contract.

From \autoref{eq:2.3} we have 
$$F(q_r^*)=1-\frac{w+c_r-v}{p-v+g_r}$$
It's obvious that there is a one-for-one mapping between $w$ and $q_r^*$, then we have $$w(q)=(p-v+g_r)\overline{F}(q)-(c_r-v),$$
the unique wholesale price that induces the retailer to order $q_r^*$ units.
Then we have the supplier's profit function:
\begin{equation}
    \pi_s(q,w(q))=g_s S(q)+(w(q)-c_s)q-g_s\mu\label{eq:2.4},
\end{equation}
from this we know that the \textit{compliance regime} \textbf{does not} matter with this contract: for a fixed $w$ no less than $c_s$ the supplier's profit  is nondecreasing in $q$.

We have the supplier's marginal profit:
\begin{align*}
    \frac{\partial\pi_s(q,w(q))}{\partial q}&=g_s S^\prime(q)+w(q)-c_s+w^\prime(q)q\\
    &=(p-v+g_r)\overline{F}(q)\left(1+\frac{g_s}{p-v+g_r}-\frac{q f(q)}{\overline{F}(q)}\right)-(c-v)
\end{align*}
$\pi_s(q,w(q))$ is decreasing in $q$ if $qf(q)/\overline{F}(q)$ is increasing. This type of demand distributions are called increasing generalized failure rate (\textbf{IGFR}) distributions.

Similarly, from \ref{eq:retailer} we have 
\begin{align*}
    \pi_r(q,w(q))&=(p-v+g_r)S(q)-(c_r-v)q-g_rv-w(q)q\\
    &=(p-v+g_r)\left(S(q)-\oF q\right)-g_rv
\end{align*}
then we have 
\begin{equation*}
    \frac{\partial\pi_r(q,w(q))}{\partial q}=(p-v+g_r)f(q)q>0,
\end{equation*}
so the supplier can increase the retailer's profit by reducing the price. 
The supply chain's profit is increasing in $q$ for $[q_s^*,q^o]$ and so is the retailer's profit. Hence, \textbf{an increase in retail power can actually improve supply chain performance.} 

Define the efficiency of the contract, $\Pi(q_s^*)/\Pi(q^o)$ and $\pi_s(q_s^*,w(q_s^*))/\Pi(q_s^*)$, the supplier's profit share. For a broad set of demand distributions, the argument that the retailer is being compensated for \textbf{the risk that demand and supply do no match} holds, where both measures approach 1 with the variation approach 0.\cite{lariviere_selling_2001}

Two-period version of the model which has excess inventory and demand updating. \textit{Push} and \textit{pull} strategies. Advanced purchase discount $w_1<w_2$. The supply chain effciency is substantially higher. There exist conditions in which advanced purchase discounts coordinate the supply chain and arbitrarily allocate its profit. \textbf{TBD}

\subsection{The buyback contract}
With a buyback contract the supplier charges the retailer $w$ per unit puchased, but pays the retailer $b$ per unit remaining at the end of the season:
$$T_b(q,w,b)=wq-bI(q)=b S(q)+(w-b)q.$$
See \cite{pasternack_optimal_1985} for detail. An important \textbf{implicit} assumption is that the supplier is able to verify the number of remaining units and the cost of such monitoring does not negate the benefits created by the contract.

The retailer's profit now is:
$$\pi_{\mathrm{r}}\left(q, w_{\mathrm{b}}, b\right)=\left(p-v+g_{\mathrm{r}}-b\right) S(q)-\left(w_{\mathrm{b}}-b+c_{\mathrm{r}}-v\right) q-g_{\mathrm{r}} \mu$$
Consider $\{w_b,b\}$ such that for $\lambda\geq 0$,
\begin{align}
    &p-v+g_r-b=\lambda(p-v+g)\label{eq:2.5}\\
    &w_b-b+c_r-v=\lambda(c-v)\label{eq:2.6}
\end{align}
A Comparing with \autoref{eq:2.1} leads to:
\begin{align}
    \pi_r(q,w_b,b)&=\lambda(p-v+g)S(q)-\lambda(c-v)q-g_r\mu\nonumber\\
    &=\lambda\Pi(q)+\mu(\lambda g-g_r).\label{eq:2.7}
\end{align}
The supplier's profit function is 
\begin{equation*}
    \pi_s(q,w_b,b)=(1-\lambda)\Pi(q)-\mu(\lambda g-g_r).
\end{equation*}
So the buyback contract \textbf{coordinates} with voluntary compliance as long as $\lambda\leq 1$. When $\lambda=1$ (or $\lambda=0$), the $q^o$ is optimal for the supplier (or retailer), but so is every other quantity since the profit function is not related with $q$. Hence, coordination is possible but no longer the uniuqe Nash equilibrium.

The $\lambda$ parameter acts to allocate the supply chain's profit between the two firms. The retailer earns the entire supply chain profit $\pi_r(q^o,w_b,b)=\Pi(q^o)$ when 
\begin{equation}\label{eq:2.8}
    \lambda=\frac{\Pi(q^o)+\mu g_r}{\Pi(q^o)+\mu g}\leq 1
\end{equation}
and the supplier $\pi_s(q^o,w_b,b)=\Pi(q^o)$, when
\begin{equation}\label{eq:2.9}
    0\leq \lambda=\frac{\mu g_r}{\Pi(q^o)+\mu g}.
\end{equation}
So \textbf{every} possible profit allocation is feasible with this set of coordinating contracts, assuming $\lambda=0$ and $\lambda=1$ are considered feasible.

\begin{note}
    The coordination of the supply chain requires the \textbf{simultaneous adjustment} of both the wholesale price $w_b$ and the buyback rate $b$. This has implications for the bargaining process, e.g., never negotiate those parameters sequentially.
\end{note}

\begin{note}
    Stock rebalancing in centralized system and decentralized system.
\end{note}

\subsection{The revenue-sharing contract}
With a revenue-sharing contract the supplier charges $w_r$ per unit purchased plus the retailer gives the supplier a percentage of his revenue. Let $\phi$ be the fraction of revenue that retailer keeps.

The transfer payment with revenue sharing is 
\begin{align*}
    T_r(q,w_r,\phi)&=w_r q+(1-\phi)(vI(q)+pS(q))\\
    &=(w_r+(1-\phi)v)q+(1-\phi)(p-v)S(q)
\end{align*}
The retailer's profit function is
$$
\pi_{\mathrm{r}}\left(q, w_{\mathrm{r}}, \phi\right)=\left(\phi(p-v)+g_{\mathrm{r}}\right) S(q)-\left(w_{\mathrm{r}}+c_{\mathrm{r}}-\phi v\right) q-g_{\mathrm{r}} \mu
$$
Now consider the set of revenue-sharing contracts, $\left\{w_{\mathrm{r}}, \phi\right\}$, such that $\lambda \geq 0$ and
$$
\begin{aligned}
&\phi(p-v)+g_{\mathrm{r}}=\lambda(p-v+g) \\
&w_{\mathrm{r}}+c_{\mathrm{r}}-\phi v=\lambda(c-v)
\end{aligned}
$$
Now we have 
\begin{align}
    &\pi_r(q,w_r,\phi)=\lambda\Pi(q)+\mu(\lambda g - g_r)\label{eq:2.10}\\
    &\pi_s(q,w_r,\phi)=(1-\lambda)\Pi(q)-\mu(\lambda g-g_r)\nonumber.
\end{align}
It's obvious that \autoref{eq:2.8} and \autoref{eq:2.9} provides the same $\lambda$.

From \autoref{eq:2.10} and \autoref{eq:2.7} we find similarity. Consider a coordinating buyback contract $\{w_b,b\}$. The retailer pays $w_b-b$ for each unit purchased and an additional $b$ per unit sold. With revenue sharing the retailer pays $w_r+(1-\phi)v$ and $(1-\phi)(p-v)$.  Now they are equivalent when 
\begin{align*}
    w_b-b&=w_r+(1-\phi)v\\
    b&=(1-\phi)(p-v)
\end{align*}
\begin{note}
    Their path will diverge in more complex settings.
\end{note}

\subsection{The quantity-flexibility contract}
With a quantity-flexibility contract, the supplier charges $w_q$ per unit purchased but then compensates the retailer for his losses on unsold units. The retailer receives a credit from the supplier at the end of the season equal to $(w_q+c_r-v)\min(I,\delta q)$, where $I$ is the leftover and $\delta\in[0,1]$ a contract parameter. It \textbf{fully} protects the retailer on \textbf{a portion of} the retailer's order whereas the buyback contract gives \textbf{partial} protection on the retailer's \textbf{entire order}. 
% Tsay 1999.

Now the transfer payment is 
% Actually, I don't how to derive this equation.
\begin{equation*}
    T_q(q,w_q,\delta)=w_q q-(w_q+c_r-v)\int_{(1-\delta)q}^q F(y)dy
\end{equation*}
\begin{note}
    Need to be checked.
\end{note}
The retailer's profit function is 
$$\begin{aligned}
    \pi_{\mathrm{r}}\left(q, w_{q}, \delta\right)=&\left(p-v+g_{\mathrm{r}}\right) S(q)-\left(c_{\mathrm{r}}-v\right) q-T_{q}\left(q, w_{q}, \delta\right)-\mu g_{\mathrm{r}} \\
    =&\left(p-v+g_{\mathrm{r}}\right) S(q)-\left(w_{q}+c_{\mathrm{r}}-v\right) q \\
    &+\left(w_{q}+c_{\mathrm{r}}-v\right) \int_{(1-\delta) q}^{q} F(y) \mathrm{d} y-\mu g_{\mathrm{r}}.
\end{aligned}$$
To achieve supply chain coordination it is necessary that
\begin{equation}\label{eq:2.11}
    (p-v+g_r)S^\prime(q^o)-(w_q+c_r-v)\left(1-F(q^o)+(1-\delta)F((1-\delta)q^o)\right)=0.
\end{equation}
Let $w_{q}(\delta)$ be the wholesale price that satisfies \autoref{eq:2.11}:
$$
w_{q}(\delta)=\frac{\left(p-v+g_{\mathrm{r}}\right)\left(1-F\left(q^{\mathrm{o}}\right)\right)}{1-F\left(q^{\mathrm{o}}\right)+(1-\delta) F\left((1-\delta) q^{\mathrm{o}}\right)}-c_{\mathrm{r}}+v
$$
$w_{q}(\delta)$ is indeed a coordinating wholesale price if the retailer's profit function is concave:
$$
\begin{aligned}
\frac{\partial^{2} \pi_{\mathrm{r}}\left(q, w_{q}(\delta), \delta\right)}{\partial q^{2}} &=-\left(p+g_{\mathrm{r}}-w_{q}(\delta)-c_{\mathrm{r}}\right) f(q)-\left(w_{q}(\delta)+c_{\mathrm{r}}-v\right)\left(1+(1-\delta)^{2} f((1-\delta) q)\right) \\
& \leq 0
\end{aligned}
$$
which holds when $v-c_{\mathrm{r}} \leq w_{q}(\delta) \leq p+g_{\mathrm{r}}-c_{\mathrm{r}}$. That range is satisfied with $\delta \in[0,1]$ because
$$
\begin{aligned}
&w_{q}(0)=\left(p-v+g_{\mathrm{r}}\right) \bar{F}\left(q^{\mathrm{o}}\right)+v-c_{\mathrm{r}} \\
&w_{q}(1)=p+g_{\mathrm{r}}-c_{\mathrm{r}}
\end{aligned}
$$
and $w_{q}(\delta)$ is increasing in $\delta$.

Now we consider supplier's profit function:
$$
\pi_{\mathrm{s}}\left(q, w_{q}(\delta), \delta\right)=g_{\mathrm{s}} S(q)+\left(w_{q}(\delta)-c_{\mathrm{s}}\right) q-\left(w_{q}(\delta)+c_{\mathrm{r}}-v\right) \int_{(1-\delta) q}^{q} F(y) \mathrm{d} y-\mu g_{\mathrm{s}}
$$
and
$$
\begin{aligned}
\frac{\partial \pi_{\mathrm{s}}\left(q, w_{q}(\delta), \delta\right)}{\partial q}=& g_{\mathrm{s}}(1-F(q))+\left(w_{q}(\delta)-c_{\mathrm{s}}\right)-\left(w_{q}(\delta)+c_{\mathrm{r}}-v\right)(F(q)\\
&-(1-\delta) F((1-\delta) q)) \\
=& g_{\mathrm{s}}(1-F(q))-c+v+\left(w_{q}(\delta)+c_{\mathrm{r}}-v\right)(1-F(q)\\
&+(1-\delta) F((1-\delta) q))
\end{aligned}
$$
The supplier's first-order condition at $q^{\circ}$ is satisfied:
$$
\frac{\partial \pi_{\mathrm{s}}\left(q^{\mathrm{o}}, w_{q}(\delta), \delta\right)}{\partial q}=g_{\mathrm{s}}\left(1-F\left(q^{\mathrm{o}}\right)\right)-c+v+\left(p-v+g_{\mathrm{r}}\right)\left(1-F\left(q^{\mathrm{o}}\right)\right)=0
$$
\begin{note}
    See \autoref{eq:2.2}
\end{note}
However, the sign of the second-order condition at $q^{\circ}$ is ambiguous,
$$
\frac{\partial^{2} \pi_{\mathrm{s}}\left(q, w_{q}(\delta), \delta\right)}{\partial q^{2}}=-w_{q}(\delta)\left(f(q)-(1-\delta)^{2} f((1-\delta) q)\right)-g_{\mathrm{s}} f(q)
$$
Hence, supply chain coordination under \textbf{voluntary compliance} is \textbf{not assured} with a quantity-flexibility contract even if the wholesale price is $w_q(\delta)$. It's \textbf{achieved} under \textbf{forced compliance} since then the supplier's action is \textbf{not relevant}.
\begin{note}
    There are some conditions that makes $q^o$ a local maximum, e.g., $\mu=10$, $\sigma=1$, $p=10$, $c_s=1$, $c_r=0$, $g_r=g_s=v=0$ and $\delta=0.1$. 
\end{note}

\textbf{Assuming a $(w_q(\delta),\delta)$ quantity-flexibility contract coordinates the channel.} When $\delta=0$, for the retailer we have
\begin{align*}
    \pi_r(q,w_q(0),0)&=(p-v+g_r)S(q)-\left(\frac{p-v+g_r}{p-v+g}\right)(c-v)q^o-\mu g_r\\
    &=\Pi(q^o)+g_s\left(\mu-S(q^o)+\overline{F}(q^o) q^o \right)\\
    &\geq \Pi(q_o)
\end{align*}
When $\delta=1$, for the supplier we have
\begin{align*}
    \pi_s(q,w_q(1),1)&=g_s S(q^o)+(p+g_r-c)q^o-(p+g_r-v)\int_0^q F(y)dy-\mu g_s\\
    &=\Pi(q^o)+\mu g_r\\
    &\geq \Pi(q^o)
\end{align*}
Since the profit function is continuous in $\delta$, all possible allocation of $\Pi(q^o)$ are possible.

\subsection{The sales-rebate contract}
% \begin{note}
%     \cite{taylor_supply_2002} and \cite{krishnan_coordinating_2004}
% \end{note}
With a sales-rebate contract the supplier charges ws per unit purchased
but then gives the retailer an r rebate per unit sold above a threshold t.
The transfer payment with the sales-rebate contract is
$$
T_{\mathrm{s}}\left(q, w_{\mathrm{s}}, r, t\right)= \begin{cases}w_{\mathrm{s}} q & q<t \\ \left(w_{\mathrm{s}}-r\right) q+r\left(t+\int_{t}^{q} F(y) \mathrm{d} y\right) & q \geq t\end{cases}
$$
\begin{note}
    $T=w_s q-r E[\left(\min(q,D)-t\right)^+]$
\end{note}
For this contract to achieve supply chain coordination, $q^{o}$ must at least be a local maximum:
\begin{equation}\label{eq:2.12}
    \frac{\partial \pi_{\mathrm{r}}\left(q^{\mathrm{o}}, w_{\mathrm{s}}, r, t\right)}{\partial q}=\left(p-v+g_{\mathrm{r}}\right) \bar{F}\left(q^{\mathrm{o}}\right)-\left(c_{\mathrm{r}}-v\right)-\frac{\partial T_{\mathrm{s}}\left(q^{\mathrm{o}}, w_{\mathrm{s}}, r, t\right)}{\partial q}=0
\end{equation}
If $q^o\leq t$, the above leads to $w_s=c_s-g_s\overline{F}(q^o)\leq c_s$, which is not acceptable to the supplier. So assume $q^o>t$. Then from \autoref{eq:2.12} we have 
\begin{equation}\label{eq:2.13}
    w_s(r)=(p-v+g_r+r)\overline{F}(q^o)-c_r+v
\end{equation}
Thus, we have the retailer's profit function
$$
\begin{aligned}
\pi_{\mathrm{r}}\left(q, w_{\mathrm{s}}(r), r, t\right)=& \Pi(q)+g_{\mathrm{s}}\left(\mu-S(q)+q \bar{F}\left(q^{\mathrm{o}}\right)\right)-r q \bar{F}\left(q^{\mathrm{o}}\right) \\
+& \begin{cases}0 & q<t \\
r q-r\left(t+\int_{t}^{q} F(y) \mathrm{d} y\right) & q \geq t\end{cases}
\end{aligned}
$$
and
$$
\begin{aligned}
\pi_{\mathrm{r}}\left(q^{\mathrm{o}}, w_{\mathrm{s}}(r), r, t\right)=& \Pi\left(q^{\mathrm{o}}\right)+g_{\mathrm{s}}\left(\mu-S\left(q^{\mathrm{o}}\right)+q^{\mathrm{o}} \bar{F}\left(q^{\mathrm{o}}\right)\right) \\
&+r\left(q^{\mathrm{o}} F\left(q^{\mathrm{o}}\right)-t-\int_{t}^{q^{\mathrm{o}}} F(y) \mathrm{d} y\right)
\end{aligned}
$$
With $t=0$ the retailer earns more than $\Pi\left(q^{\circ}\right)$, so $q^{\circ}$ is surely optimal. With $t=q^{\mathrm{o}}$, the retailer's profit function is decreasing for $t \geq q^{\mathrm{o}} ; \bar{q}$ is at least as good for the retailer as $q^{\mathrm{o}}$. Given that $\pi_{\mathrm{r}}\left(q^{\mathrm{o}}, w_{\mathrm{s}}(r), r, t\right)$ is decreasing in $t$, there must exist some $t$ in the range $\left[0, q^{\mathrm{o}}\right]$ such that $\pi_{\mathrm{r}}\left(q^{\mathrm{o}}, w_{\mathrm{s}}(r), r, t\right)=$ $\pi_{\mathrm{r}}\left(\bar{q}, w_{\mathrm{s}}(r), r, t\right)$, i.e., there are coordinating contracts such that $q^o$ is preferred by the retailer over $\overline{q}$.
\begin{note}
    Why there must exist some $t$ in $[0,q^o]$ such that $\pi_{\mathrm{r}}\left(q^{\mathrm{o}}, w_{\mathrm{s}}(r), r, t\right)=$ $\pi_{\mathrm{r}}\left(\bar{q}, w_{\mathrm{s}}(r), r, t\right)$?
\end{note}
\begin{note}
    It's easy to check there are a set of contracts that generate any allocation of supply chain's profit.
\end{note}
Now consider the supplier's production decision. The supplier's profit function in this type of contract is 
$$
\pi_{\mathrm{s}}\left(q, w_{\mathrm{s}}(r), r, t\right)=-g_{\mathrm{s}}(\mu-S(q))-c_{\mathrm{s}} q+T_{\mathrm{s}}\left(q, w_{\mathrm{s}}(r), r, t\right)
$$
For $q>t$
$$
\begin{aligned}
\frac{\partial \pi_{\mathrm{s}}\left(q, w_{\mathrm{s}}(r), r, t\right)}{\partial q} &=g_{\mathrm{s}} \bar{F}(q)-c_{\mathrm{s}}+w_{\mathrm{s}}(r)-r+r F(q) \\
&=\left(r-g_{\mathrm{s}}\right)\left(F(q)-F\left(q^{\mathrm{o}}\right)\right)
\end{aligned}
$$
To have $q^o$ a local maximum for the supplier, we should have $r<g_s$ for $q\leq q^o$ which leads to that $w_s(r)\leq c_s$ and the supplier cannot earn a positive profit. Thus we must have $r>g_s$, but this implies the supplier loss money for each unit delivered to the retailer above $t$ by \autoref{eq:2.13}:
\begin{equation*}
    w_s(r)-r=c_s-v-g_s\overline{F}(q^o)-rF(q^o)<c_s.
\end{equation*}
Thus the sales-rebate contract does not coordinate the supply chain with voluntary compliance.

\subsection{The quantity-discount contract}
This section considers an "all unit" quantity discount, i.e., $T_d(q)=w_d(q)q$ where $w_d(q)$ is decreasing in $q$.
\begin{note}
    There are many types of quantity discounts. See Moorthy (1987) for a more detailed explanation for why many coordinating quantity discount schedules exist. See Kolay and Shaffer (2002) for a discussion on different types of quantity discounts. 
\end{note}
The retailer's profit function is then 
$$\pi_r(q,w_d(q))=(p-v+g_r)S(q)-(w_d(q)+c_r-v)q-g_r\mu.$$
One technique to obtain coordination is to choose the payment schedule such that the retailer's profit equals a constant fraction of the supply chain's profit. To be specific, let 
$$w_d(q)=((1-\lambda)(p-v+g)-g_s)\left(\frac{S(q)}{q}\right)+\lambda(c-v)-c_r+v.$$
The above is decreasing in $q$ if $\lambda\leq\overline{\lambda}$, where 
$$\overline{\lambda}=\frac{p-v+g_r}{p-v+g},$$
since $S(q)/q$ is decreasing in $q$. The retailer's profit function is now 
\begin{align*}
    \pi_r(q,w_d(q))&=\lambda(p-v+g)S(q)-\lambda(c-v)q-g_r\mu\\
    &=\lambda(\Pi(q)+g\mu)-g_r\mu
\end{align*}
Hence $q^o$ is optimal for both the retailer and the supplier. The parameter $\lambda$ acts to allocate the supply chain's profit between the two firms, however, it has an upperbound that prevents too much profit from being allocated to the retailer. The $w_d(q)$ can still coordinate even if $\lambda>\overline{\lambda}$ but then the $w_d(q)$ will be increasing in $q$, i.e., a quantity-premium contract.

\subsection{Discussion}
Revenue sharing and quantity discounts always coordinate the supplier's action with voluntary compliance, quantity-flexibility contracts generally, but not always, coordinate the supplier's action and sales-rebate contracts never do.

he coordinating revenue-sharing contracts do not depend on the demand distribution, but do depend on the retailer's marginal cost.
\begin{note}
    \textbf{Read} Inducing Forecast Revelation through Restricted Returns
\end{note}




\newpage
\numberwithin{equation}{section}
\newcommand{\spq}{S(q,p^o(q))}
\newcommand{\pq}{p^o(q)}
\thispagestyle{empty}
\section{Coordinating the newsvendor with \textit{price-dependent} demand}


\subsection{Model and analysis}
Now the retailer chooses his price in addition to his order quantity. Let $F(q|p)$ be the distribution function of demand, where $p$ is the retail price. Assume $\frac{\partial F(q|p)}{\partial p}>0$. To obtain initial insights, assume the retailer sets his \textbf{price} \textit{at the same time as} his \textbf{stocking decision} and the price is \textbf{fixed} throughout the season.
\begin{note}
    van Mieghem and Dada (1999). A hybrid model. The retailer chooses $q$, then observes a demand signal and then chooses price.
\end{note}

The integrated channel's profit is 
\begin{equation*}
    \Pi(q,p)=(p-v+g)S(q,p)-(c-v)q-g\mu
\end{equation*}
where $S(q,p)$ is expected sales given the stocking quantity $q$ and the price $p$, and similarly, we have
\begin{equation*}
    S(q,p)=q-\int_0^q F(y|p)dy
\end{equation*}
\begin{note}
    The integrated channel profit function need not be concave nor unimodal (Petruzzi \& Dada 1999)
\end{note}
Let $\pq$ be the supply chain optimal price for a given $q$. The necessary condition for coordination is
\begin{equation}\label{eq:3.1}
    \frac{\partial\Pi(q,p^o(q))}{\partial p}=S(q,p^o(q))+(p^o(q)-v+g)\frac{\partial S(q,p^o(q))}{\partial p}=0.
\end{equation}
\begin{note}
    Either not satisfy the first-order condition or fail to coordinate the quantity decision.
\end{note}

Consider the \textbf{quantity-flexibility} contract. The retailer's profit function is 
\begin{align*}
    \pi_r(q,p,w_q,\delta)=&(p-v+g_r)S(q,p)-(w_q+c_r-v)q\\
    +&(w_q+c_r-v)\int_{(1-\delta)q}^q F(y|p)dy-\mu g_r
\end{align*}
For price coordination the first-order condition must hold,

\begin{align}
\frac{\partial \pi_{\mathrm{r}}\left(q, p^{\mathrm{o}}(q), w_{q}, \delta\right)}{\partial p}=& S\left(q, p^{\mathrm{o}}(q)\right)+\left(p^{\mathrm{o}}(q)-v+g_{\mathrm{r}}\right) \frac{\partial S\left(q, p^{\mathrm{o}}(q)\right)}{\partial p} \nonumber\\
&+\left(w_{q}+c_{\mathrm{r}}-v\right) \int_{(1-\delta) q}^{q} \frac{\partial F\left(y \mid p^{\mathrm{o}}(q)\right)}{\partial p} \mathrm{~d} y \nonumber\\
=& 0\label{eq:3.2}
\end{align}
The second term in \autoref{eq:3.2} is no smaller than the second term in \autoref{eq:3.1}\footnote{The assumption of $\partial F(q|p)/\partial p>0.$}, so the above holds only if the third term is nonpositive. 
But the third term is nonnegative as $w_q+c_r-v\geq 0$, so with a coordinating $w_q$, the coordination of price can only occur if $g_s=0$ and either $w_q=v-c_r$ or $\delta=0$. \textbf{Neither} is desirable. With $w_q=v-c_r$, then supplier has $w_q<c_s$\footnote{Why? An assumption?} which is not acceptable. With $\delta=0$ the contract degenerates to just a wholesale-price contract, so the retailer's quantity action is not optimal. Hence, the quantity-flexibility contract does not coordinate the newsvendor with price-dependent demand.

The \textbf{sales-rebate} contract does not fare better:
\begin{align*}
    \frac{\partial\pi_r(q,\pq,w_s,r,t)}{\partial p}=&\spq+(\pq-v+g_r)\frac{\partial\spq}{\partial p}\\
    &-r\int_t^q\frac{\partial F(y|\pq)}{\partial p}dy
\end{align*}
Since the last term is negative when $r>0$ and $t<q$, we know that the retailer prices below the optimal price\footnote{The above derivative is negative. Why it means that the retailer prices below the optimal price?}. Coordination might be achieved if there is something to induce the retailer to a higher price.

Now consider a \textbf{buyback} contract. The retailer's profit function is 
\begin{equation*}
    \pi_r(q,p,w_b,b)=(p-v+g_r-b)S(q,p)-(w_b-b+c_r-v)q-g_r\mu.
\end{equation*}
For coordination we must have the first-order condition:
\begin{equation}\label{eq:3.3}
    \frac{\partial\pi_r(q,\pq,w_b,r,t)}{\partial p}=\spq+(\pq-v+g_r-b)\frac{\partial\spq}{\partial p}=0.
\end{equation}
But comparing with \autoref{eq:3.1} it holds only if $b=-g_s< 0$ which violates that $b\geq 0$\footnote{If $g_s=0$, then $w_b=c_s$ and $b_s=0$ which means that the supplier earns no positive profit.\label{ft:4}}. Therefore, a buyback contract does not coordinate the newsvendor with price-dependent demand.

The buyback contract fails to coordinate in this setting because the parameters of the coordinating contracts depend on the price: from \autoref{eq:2.5} and \autoref{eq:2.6}, the coordinating parameters are
\begin{align*}
    b&=(1-\lambda)(p-v+g)-g_s\\
    w_b&=\lambda c_s+(1-\lambda)(p+g-c_r)-g_s.
\end{align*}
For a fixed $\lambda$, the coordianting buyback rate and wholesale price are linear in $p$. Hence, the buyback contract coordiantes the newsvendor with price-dependent demand if $b$ and $w_b$ are made \textbf{contingent} on the retail price chosen, or if $b$ and $w_b$ are chosen \textbf{after} the retailer commits to a price (but before the retailer chooses $q$). This is the \textbf{price-discount-sharing} contract\footnote{Bernstein and Federgruen (2000)}, which is called a "bill back" in practice. The retailer gets a lower wholesale price if the retailer reduces his price, i.e., the supplier shares in the cost of a price discount with the retailer. Then we have the retailer profit function:
\begin{align*}
    \pi_r(q,p,w_b,b)&=\lambda(p-v+g)S(q,p)-\lambda(c-v)q-g_r\mu\\
    &=\lambda(\Pi(q,p)+g\mu)-g_r\mu
\end{align*}
Hence, for the retailer as well ass the supplier, $\{q^o,p^o\}$ is optimal for $\lambda\in[0,1]$.

Now consider the \textbf{revenue-sharing} contract. The retailer's profit is 
\begin{equation*}
    \pi_r(q,p,w_r,\phi)=(\phi(p-v)+g_r)S(q,p)-(w_r+c_r-\phi v)q-g_r\mu.
\end{equation*}
Coordination require
\begin{equation}\label{eq:3.4}
    \frac{\pi_r(q,\pq,w_r,\phi)}{\partial p}=\spq+(\pq-v+g_r/\phi)\frac{\partial\spq}{\partial p}=0.
\end{equation}

\begin{itemize}
    \item Consider $g_r=g_s=0$. In this situation,
    \begin{equation*}
        \frac{\partial\pi_r(q,p,w_r,\phi)}{\partial p}=\frac{\partial\Pi(q,p)}{\partial p}
    \end{equation*}
    with \textbf{any} revenue-sharing contract. Thus, the retailer chooses $\pq$ no matter which revenue-sharing contract is chosen. Now revenue sharing is able to coordinate the retailer's quantity decision with precisely the same set of contracts used when the retailer prices is fixed. 

    Recall that with the \textit{fixed price} newsvendor \textbf{revenue sharing} and \textbf{buybacks} are equivalent. Here, the contracts produce different outcomes because with a buyback the retailer's share of revenue $(1-b/p)$ depends on the price, whereas with revenue sharing it is independent of the price, by definition\footnote{The above partial derivative}. However, the \textbf{price contingent buyback} contract (\textbf{price-discount} contract) is equivalent to revenue sharing: if $g_r=g_s=0$, the coordinating revenue-sharing contract yield 
    \begin{equation*}
        \pi_r(q,p,w_r,\phi)=\lambda\Pi(q,p)
    \end{equation*}
    from \autoref{eq:2.10}. And the price contingent buyback contract yield the same profit for any quantity and price from \autoref{eq:2.7},
    \begin{equation*}
        \pi_r(q,p,b(p),w_b(p))=\lambda\Pi(q,p).
    \end{equation*}
    \item Consider at least one of $g_r$ or $g_s$ is larger than $0$. From \autoref{eq:3.4} coordination is achieved only if $\phi=g_r/g$. In this contract both firms \textbf{may} enjoy a positive profit\footnote{Only if $g_r,g_s>0$ then both firms will earn positive profits.}, which contrasts with the single coordination outcome of the buyback contract shown in \autoref{ft:4}. The difficulty with coordination occurs because the coordinating parameters generally depend on the retail price
    \begin{align*}
        \phi&=\lambda+\frac{\lambda g-g_r}{p-v},\\
        w_r&=\lambda(c-v)-c_r+\phi v.
    \end{align*}
    The \textbf{dependence} on the retail price is eliminated only in the special case $\phi=\lambda=g_r/g$.
\end{itemize}

Coordination for all profit allocations is restored even in this case if, like with the buyback contract, the parameters of the revenue-sharing contract are made contingent on the retailer's price. In that case revenue sharing is again equivalent to the price-discount contract: price discounts are contingent buybacks and contingent buybacks are equivalent to contingent revenue sharing.

Consider the final \textbf{quantity discount} contract. The retailer's profit function is 
\begin{equation*}
    \pi_r(q,w_d(q),p)=(p-v+g_r)S(q,p)-(w_d(q)+c_r-v)q-g_r\mu.
\end{equation*}
If $g_s=0$, then 
\begin{equation*}
    \frac{\pi_r(q,w_d(q),p)}{\partial p}=\frac{\partial S(q,p)}{\partial p}+(p-v+g_r)S(q,p)=\frac{\partial \Pi(q,p)}{\partial p}
\end{equation*}
and so $\pq$ is optimal for the retailer. On the otehr hand, if $g_s>0$, then the retailer's pricing decision needs to be distorted for coordination, which the quantity discount does not do.

Assuming $g_s=0$, we still need to check if the quantity is coordinated. Assume that the optimal price is chosen, we have 
\begin{equation*}
    w_d(q)=((1-\lambda)(p^o-v+g)-g_s)\frac{S(q,p^o)}{q}+\lambda(c-v)-c_r+v,
\end{equation*}
where $p^o=\pq$. It follows that 
\begin{align*}
    \pi_r(q,w_d(q),p)=(p-v+g_r)S(q,p)-\lambda(c-v)q-g_r\mu-((1-\lambda)(p^o-v+g)-g_s)S(q,p^o)
\end{align*}
and so $p^o$ is optimal for the retailer\footnote{Notice that $g_s=0$}, 
\begin{equation*}
    \frac{\pi_r(q,w_d(q),p)}{\partial p}=\frac{\partial\Pi(q,p)}{\partial p}.
\end{equation*}
Given $p^o$ is chosen,
\begin{align*}
    \pi_r(q,w_d(q),p^o)&=\lambda(p^o-v+g)S(q,p^o)-\lambda(c-v)q-g_r\mu\\
    &=(\Pi(q,p^o)+g\mu)-g_r\mu
\end{align*}
and so $q^o$ is optimal for the retailer and the supplier. Coordination occurs becasue the retailer's pricing decision is not distorted, and the retailer's quantity decision is adjusted \textbf{contingent} that $p^o$ is chosen.

\subsection{Discussion}
There are surely many situations in which a retailer has some control over his pricing. However, \textbf{incentives to coordinate the retailer's quantity decision may distort the retailer's price decision}. This occurs with the \textbf{buyback}, \textbf{quantity-flexibility} and the \textbf{sales-rebate} contracts. Since the \textbf{quantity discount} leaves all revenue with the retailer, it does not create such a distortion, which is an asset when the retailer's pricing decision should not be distorted, i.e., when $g_s=0$. \textbf{Revenue sharing} does not distort the retailer's pricing decision when $g_r=g_s=0$. In those situations the set of revenue-sharing contracts to coordinate the quantity decision with a fixed price continue to coordinate the quantity decision with a variable price. However, when there are goodwill costs, then the coordinating revenue-sharing parameters generally depend on the retail price. The dependence is removed with only a single revenue-sharing contract; hence coordination is only achieved with a single profit allocation\footnote{$\phi=g_r/g$.}. Coordination is restored with arbitrary profit allocation by making the parameters contingent on the retail price chosen, e.g., a menu of revenue-sharing contracts is offered that depend on the price selected. This technique also applies to the buyback contract: the \textbf{price contingent buyback} contract, which is also called a price-discount-sharing contract, coordinates the price-setting newsvendor. In fact, just as buybacks and revenue sharing are equivalent with a fixed retail price, the \textbf{price contingent buybac} and \textbf{revenue sharing} are equivalent when there are no goodwill costs. When there are goodwill costs then the \textbf{price contingent buy back} is equivalent to the \textbf{price contingen revenue-sharing}  contract.


















\newpage
\section{Coordinating the newsvendor with effort-dependent demand}
\begin{note}
    Netessine and Rudi (2000a)
    Wang and Gerchak 2001
    Gilbert and Cvsa 2000
\end{note}
Only the quantity-discount contract can coordinate a retailer that chooses quantity, price and effort.

\subsection{Model and analysis}
\begin{itemize}
    \item Suppose a single effort level $e$, summarizes the retailer's activities and let $g(e)$ be the retailer's cost of exerting effort level $e$, where $g(0)=0$, $g^\prime(e)>0$ and $g^{\prime\prime}(e)>0$.
    \item  Assume there are no goodwill costs, $g_r=g_s=0$, $v=0$ and $c_r=0$. Let $F(q|e)$ be the distribution of demand given the effort level $e$, where demand is \textbf{stochastically increasing in effort}, i.e., $\partial F(q|e)/\partial e<0$.
    \item Suppose the retailer chooses his effort level \textbf{at the same time as} his order quantity. 
    \item Assume the supplier \textbf{cannot verify} the retailer's effort level, which implies the retailer cannot sign a contract binding the retailer to choose a particular effort level.
\end{itemize}

Then we have 
\begin{equation*}
    \Pi(q,e)=p S(q,e)-c q-g(e),
\end{equation*}
where 
\begin{equation*}
    S(q,e)=q-\int_0^q F(y|e)dy.
\end{equation*}
\textbf{The integrated channel’s profit function need not be concave nor unimodal.}
 Assume that the integrated channel solution is well behaved, i.e., $\Pi(q,e)$ is unimodal and maximized with finite arguments. $q^o$ and $e^o$ are the optimal solutions.

 $e^o(q)$ maximizes the supplyu chain's revenue net effort cost only if 
 \begin{equation}
     \frac{\partial\Pi(q,e^o(q))}{\partial e}=p\frac{\partial S(q,e^o(q))}{\partial e}-g^\prime(e^o(q))=0.
 \end{equation}
 With a \textbf{buyback contract} the retailer's profit function is
 $$
 \pi_{\mathrm{r}}\left(q, e, w_{\mathrm{b}}, b\right)=(p-b) S(q, e)-\left(w_{\mathrm{b}}-b\right) q-g(e)
 $$
 For all $b>0$ it holds that
 \begin{equation}
    \frac{\partial \pi_{\mathrm{r}}\left(q, e, w_{\mathrm{b}}, b\right)}{\partial e}<\frac{\partial \Pi(q, e)}{\partial e}
 \end{equation}
 Thus, $e^{\mathrm{o}}$ cannot be the retailer's optimal effort level when $b>0$. But $b>0$ is required to coordinate the retailer's order quantity\footnote{\autoref{eq:2.5} and $\lambda\in(0,1)$}, so it follows that the buyback contract cannot coordinate in this setting.

With a \textbf{quantity-flexibility} contract, we have 
$$
\pi_{\mathrm{r}}\left(q, e, w_{\mathrm{q}}, \delta\right)=p S(q, e)-w_{\mathrm{q}}\left(q-\int_{(1-\delta) q}^{q} F(y \mid e) \mathrm{d} y\right)-g(e) .
$$
For all $\delta>0$ (which is required to coordinate the retailer's quantity decision)
$$
\frac{\partial \pi_{\mathrm{r}}\left(q, e, w_{\mathrm{q}}, \delta\right)}{\partial e}<\frac{\partial \Pi(q, e)}{\partial e} .
$$
As a result, the retailer chooses a \textbf{lower effort} than optimal\footnote{Because the left-hand side will first approach $0$, i.e., the retailer will choose effort level lower than $e^o(q)$.}, i.e., the quantity-flexibility contract also does not coordinate the supply chain in this setting.

Also, it can be shown that \textbf{revenue-sharing} contract with $\phi<1$ has
$$\frac{\partial \pi_{\mathrm{r}}\left(q, e, w_{\mathrm{r}}, \phi\right)}{\partial e}<\frac{\partial \Pi(q, e)}{\partial e} .$$ The \textbf{sales-rebate} contract with $r>0$ and $q>t$ has
$$\frac{\partial \pi_{\mathrm{r}}\left(q, e, w_{\mathrm{s}}, r,t\right)}{\partial e}>\frac{\partial \Pi(q, e)}{\partial e},$$
which means the retailer exerts too much effort.

Consider \textbf{quantity discount} contract\footnote{The quantity discount should let the retailer retain the revenues but charge a marginal cost based on expected revenue conditional on the optimal effort.}. 
Suppose $T_d(q)=w_d(q)q$, where 
\begin{equation*}
    w_d(q)=(1-\lambda)p\left(\frac{S(q,e^o)}{q}\right)+\lambda c+(1-\lambda)\frac{g(e^o)}{q})
\end{equation*}
and $\lambda\in[0,1]$. 

Now the retailer's profit function is 
\begin{equation*}
    \pi_r(q,e)=p S(q,e)-(1-\lambda)p S(q,e^o)-\lambda c q-g(e)+(1-\lambda)g(e^o)
\end{equation*}
Given the optimal effort $e^o$, the retailer's profit function is 
$$\pi_r(q,e^o)=\lambda p S(q,e^o)-\lambda cq-\lambda g(e^o)=\lambda\Pi(q,e^o),$$
and so the retailer's optimal order quantity is $q^o$, any allocation of profit is feasible and the supplier's optimal production is $q^o$.

Also, if the demand is dependent on price and effort, let 
\begin{equation*}
    w_d(q)=(1-\lambda)p^o \left(\frac{S(q,e^o)}{q}\right)+\lambda c+(1-\lambda)\frac{g(e^o)}{q}.
\end{equation*}
Again, the retailer retains all revenue and so optimizes price and effort\footnote{The logic.}. Futhermore, the quantity decision is not distorted because the quantity-discount schedule is contingent on the optimal price and effort and not on the chosen price and effort. 



\newpage
\section{Coordination with multiple newsvendors}
This section considers two models with one supplier and multiple competing retailers. 

\subsection{Competing newsvendors with a fixed retail price}
Set $c_r=g_r=g_s=v=0$, increase the number of retailers to $n>1$. $D$ the total retail demand. And for each retailer $i$'s demand:
\begin{equation*}
    D_i=\left(\frac{q_i}{q}\right)D,
\end{equation*}
where $q=\sum_{i=1}^n q_i$ and $q_{-i}=q-q_i$. 
Given the proportional allocation rule, the integrated supply chain faces a single newsvendor problem. Hence we have 
\begin{equation}\label{eq:5.1}
    F(q^o)=\frac{p-c}{p}.
\end{equation}
Retailer $i$'s profit function with a buyback contract is 
\begin{equation*}
    \pi_i(q_i,q_{-i})=(p-w)q_i-(p-b)\left(\frac{q_i}{q}\right)\int_0^q F(x)dx.
\end{equation*}
The above also provides the retailer's profit with a wholesale-price contract (i.e., set $b=0$). It's strictly concave in $q$. Hence, for every $q_{-i}$ there is a unique optimal response. Consider a Nash equilibrium $\{q^*_i\}_{i=1}^n$, it must have
\begin{equation*}
    \frac{\partial\pi_i(q_i,q_{-i})}{\partial q_i}=q^*\left(\frac{p-w}{p-b}\right)-q^*_i F(q^*)-q_{-i}^*\left(\frac{1}{q^*}\int_0^{q^*}F(x)dx\right)=0.
\end{equation*}
Substitute $q_{-i}^{*}=q^{*}-q_{i}^{*}$ into the above equation and solve for $q_{i}^{*}$ given a fixed $q^{*}$ :
\begin{equation}\label{eq:5.2}
    q_{i}^{*}=q^{*} \frac{\left((p-w) /(p-b)-\left(1 / q^{*}\right) \int_{0}^{q^{*}} F(x) \mathrm{d} x\right)}{F\left(q^{*}\right)-1 / q^{*} \int_{0}^{q^{*}} F(x) \mathrm{d} x} .
\end{equation}
Now substitute it into $q^*=n q_i^*$, then we have
\begin{equation}\label{eq:5.3}
    g(q^*)\equiv\frac{1}{n}F(q^*)+\left(\frac{n-1}{n}\right)\left(\frac{1}{q^*}\int_0^{q^*}F(x)dx\right)=\frac{p-w}{p-b}.
\end{equation}
It's easy to see that $g(0)=1$, $g(\infty)=1$ and $g^\prime(\cdot)>0$. Thus, when $b<w<p$, there exists a unique $q^*$ satisfying \autoref{eq:5.3}.

Consider $n$. LHS in \autoref{eq:5.3} is decreasing in $n$, thus $q^*$ is increasing in $n$. Competition makes the retailers order more inventory because of the \textbf{demand-stealing effect}: each retailer \textbf{ignores} the fact that ordering more means the other retailers' demands \textbf{stochastically decrease}.

Due to the \textbf{demand-stealing effect} the supplier can coordinate the supply chain and earn a positive profit with just a wholesale-price contract.
Let $\hat{w}(q)$ be the wholesale price that induces the retailers to order $q$ units with a wholesale-price contract (i.e., with $b=0$). From \autoref{eq:5.3},
\begin{equation*}
    \hat{w}(q)=p\left(1-\left(\frac{1}{n}\right) F(q)-\left(\frac{n-1}{n}\right)\left(\frac{1}{q} \int_{0}^{q} F(x) \mathrm{d} x\right)\right).
\end{equation*}
By definition $\hat{w}(q^o)$ is the coordinating wholesale price. Given $F(q^o)=(p-c)/c$ and 
\begin{equation*}
    \frac{1}{q}\int_0^q F(x)dx<F(q),
\end{equation*}
it can be shown that $\hat{w}(q^o)>c$ when $n>1$\footnote{
    $\hat{w}(q^o)>p\left(1-F(q^o)\right)=p(1-\frac{p-c}{c})=\frac{2cp-p^2}{c}$ ?????????\textbf{TBD.}
}. Hence, the supplier earns a positive profit. But with the \textbf{single} retailer model channel coordination is only achieved when the supplier earns zero profit, i.e., $\hat{w}(q^o)=c$. 

But the coordination is not optimal for supplier. The profit function is
\begin{equation*}
    \pi_s(q,\hat{w}(q))=q(\hat{w}(q)-c).
\end{equation*}
Assuming $n>1$, we have 
\begin{equation*}
    \frac{\partial\pi_s(q^o,\hat{w}(q^o))}{\partial q}=-\frac{q^o p f(q^o)}{n}<0.
\end{equation*}
\begin{note}
    Checked but don't sure.
\end{note}
Hence, the supplier prefers to sell less than $q^o$ and charges a higher wholesale price when $n>1$. 

For supplier, a coordinating buyback contract ($w_b(b)$) may exceed the profit with the optimal wholesale-price contract. Since the buybakc rate provides an incentive to the retailers to increase their order quantity, it must be that $w_b(b)>\hat{w}(q^o)$. From \autoref{eq:5.1} and \autoref{eq:5.3} 
\begin{equation*}
    w_b(b)=p-(p-b)\left[\frac{1}{n}\left(\frac{p-c}{p}\right)+\left(\frac{n-1}{n}\right)\left(\frac{1}{q^o}\int_0^{q^o}F(x)dx\right)\right].
\end{equation*}
Given that $q_{i}^{*}=q^{*} / n$, retailer $i$ 's profit with a coordinating buyback contract is
$$
\begin{aligned}
\pi_{i}\left(q_{i}^{*}, q_{-i}^{*}\right) &=(p-w(b)) q^{\mathrm{o}} / n-(p-b)\left(\frac{1}{n}\right) \int_{0}^{q^{\circ}} F(x) \mathrm{d} x \\
&=\left(\frac{p-b}{p n^{2}}\right) q^{\mathrm{o}}\left[p-c-\frac{p}{q^{\mathrm{o}}} \int_{0}^{q^{\circ}} F(x) \mathrm{d} x\right] \\
&=\left(\frac{p-b}{p n^{2}}\right) \Pi\left(q^{\mathrm{o}}\right)
\end{aligned}
$$
The supplier's profit with the coordinating contract is
$$
\begin{aligned}
\pi_{\mathrm{s}}\left(q^{\mathrm{o}}, w_{\mathrm{b}}(b), b\right) &=\Pi\left(q^{\mathrm{o}}\right)-n \pi_{i}\left(q_{i}^{*}, q_{-i}^{*}\right) \\
&=\left(\frac{p(n-1)+b}{p n}\right) \Pi\left(q^{\mathrm{o}}\right)
\end{aligned}
$$
When $b=p$, the supplier extracts all supplier chain profit and certainly earns more than in the wholesale-price contract since in which it sells less than $q^o$. Also, we have
\begin{equation*}
    \frac{\pi_s(q^o,w_b(0),0)}{\Pi(q^o)}=\frac{n-1}{n}.
\end{equation*}
Hence, as $n$ increases the supplier's potential gain decreases from using a coordinating buyback contract rather than her optimal wholesale-price contract.

\subsection{Competing newsvendor with market-clearing prices}
In this model, the market price depends on the realization of demand and the amount of inventory purchased.

Suppose industry demand can take on high or low state. $q$ the total order quantity. We have the market-clearing prices
\begin{align*}
    p_l(q)&=(1-q)^+\\
    p_h(q)&=\left(1-\frac{q}{\theta}\right)^+
\end{align*}
for $\theta>1$. Suppose either demand is equally likely.

There is a continuum of retailers on $[0,1]$. Retailers must order inventory from a single supplier \textbf{before} the realization of the demand is observed. \textbf{After} demand is observed the market-clearing price is determined. \textbf{Perfect competition is assumed.} Leftover inventory has no salvage value and the supplier's production cost is zero. 

To set a benchmark, consider a single monopolist. We have the optimal profit
$$\Pi^o=\frac{1}{2}p_1l(
\frac{1}{2})\frac{1}{2}+\frac{1}{2}p_h(\frac{\theta}{2})\frac{\theta}{2}=\frac{1+\theta}{8}.$$

Now consider the case that the supplier sells to the perfectly competitive retailers with a wholesale-price contract. The expected profit is 
\begin{equation*}
    \frac{1}{2} p_{l}(q) q+\frac{1}{2} p_{\mathrm{h}}(q) q-w q= \begin{cases}\frac{1}{2} q(2-q-q / \theta)-w q & q \leq 1 \\ \frac{1}{2} q(1-q / \theta)-w q & q>1\end{cases}.   
\end{equation*}
Let $q_1(w)$ be the quantity that sets the above profit to zero when $q\leq 1$, which is the equilibrium outcome due to perfectly competition:
$$q_1(w)=\frac{2\theta}{1+\theta}(1-w),$$
which is hold if $w\geq (1/2)-1/(2\theta)$. Consider $q_2(w)$ when $q>1$,
$$q_2(w)=\theta(1-2w),$$
which is hold if $w<(1/2)-1/(2\theta)$.

Then we have the supplier's profit:
\begin{equation*}
    \pi_s(w)=\begin{cases}
        q_1(w)w & w\geq (1/2)-1/(2\theta)\\
        q_2(w)w&otherwise
    \end{cases}
\end{equation*}
Let $w^{*}(\theta)$ be the supplier's optimal wholesale price:
$$
w^{*}(\theta)= \begin{cases}\frac{1}{2} & \theta \leq 3 \\ \frac{1}{4} & \text {otherwise }\end{cases}
$$
and
$$
\pi_{\mathrm{s}}\left(w^{*}(\theta)\right)= \begin{cases}\frac{\theta}{2(1+\theta)} & \theta \leq 3 \\ \frac{1}{8} \theta & \text {otherwise }\end{cases}
$$
Then we have the reatilers' order. When $\theta \leq 3$ the retailers order
$$
q_{1}\left(w^{*}(\theta)\right)=\frac{\theta}{1+\theta}
$$
and the market-clearing prices are
$$
p_{l}\left(q_{1}\left(w^{*}(\theta)\right)\right)=\frac{1}{1+\theta}, \quad p_{h}\left(q_{1}\left(w^{*}(\theta)\right)\right)=\frac{\theta}{1+\theta}
$$
When $\theta>3$ the retailers order
$$
q_{2}\left(w^{*}(\theta)\right)=\frac{\theta}{2}
$$
and the market-clearing prices are
$$
p_{l}\left(q_{2}\left(w^{*}(\theta)\right)\right)=0, \quad p_{h}\left(q_{2}\left(w^{*}(\theta)\right)\right)=\frac{1}{2}
$$

No matter the value of $\theta$, $\pi_s(w^*(\theta))<\Pi^o$, so the supplier does not capture the maximum possible profit with a wholesale-price contract. In either case the problem is that competition leads the retailers to sell too much in the low demand state\footnote{$\theta> 1$, $p_l>\frac{1}{2}$ in either case.}. The monopolist does not sell all of her inventory in the low-demand state, but the perfectly competitive retailers cannot be so restrained.
\newcommand{\op}{\overline{p}}

To earn a higher profit the supplier must devise a mechanism to prevent
the low-demand state market-clearing price from falling below 1/2. Deneckere et al. (1997) propose the supplier implements resale price maintenance: the retailers \textbf{may not sell below a stipulated price}. Let $\op$ be that price. When $\op$ is above the market-clearing price the retailers have unsold inventory, so demand is allocated among the retailers\footnote{Why?}. Assume demand is allocated so that each retailer sells a constant fraction of his order quantity, i.e., proportional allocation.

Given the optimal market-clearing price is always 1/2, the search for the
optimal resale price maintenance contract should begin with $\op=1/2$. Assume the retailers' total order quantity equals $\theta/2$, i.e.,
\begin{equation}\label{eq:5.4}
    \int_0^1 q(t)dt=\frac{\theta}{2}.
\end{equation}
Hence, the market-clearing price in either demand state is $1/2$. Evaluate the $t$-th retailer's expected profit:
\begin{equation*}
    \pi_r(t)=-q(t)w+\frac{1}{2}\left(\frac{1/2}{\theta/2}q(t)\right)\op+\frac{1}{2}q(t)\op=q(t)\left(\frac{1+\theta}{4\theta}-w\right).
\end{equation*}
So the supplier can charge 
$$\overline{w}=\frac{1+\theta}{4\theta}.$$
Now show that the retailers indeed order $\theta/2$ under $\overline{w}$. 
Say the retailers order $1 / 2<q<\theta / 2$\footnote{Why?}, so the $t$-th retailer's expected profit is
$$
-q(t) w+\frac{1}{2}\left(\frac{1 / 2}{q} q(t)\right) \bar{p}+\frac{1}{2} q(t)\left(1-\frac{q}{\theta}\right) .
$$
The above is decreasing in the relevant interval and equals 0 when the wholesale price is $\bar{w}$. So with the $(\bar{p}, \bar{w})$ resale price maintenance contract the retailers order $q=\theta / 2$, the optimal quantity is sold in either state and the retailers' expected profit is zero. Hence, the supplier earns $\Pi^{o}$ with that contract.

Resale price maintenance prevents destructive competition in the low demand state, but there is another approach. Suppose the supplier offers a buyback contract with $b=1/2$ which makes the price cannot fall below $1/2$. Then the retailers' profit is 
\begin{equation*}
    \frac{1}{2}\left(p_l(1/2)(1/2)+b(q-1/2)\right)+\frac{1}{2}(p_h(q)q)-qw=q\left(\frac{3}{4}-w-\frac{q}{2\theta}\right),
\end{equation*} 
assuming $1/2<q<\theta/2$. From the above we know that the retailers earn a zero profit with $q=\theta/2$ when $w=1/2$, i.e., the supplier maximizes the system's profit.  
\begin{note}
    The supplier sets a higher wholesale price iwth the buyback contract, i.e., $\frac{1}{2}>\overline{w}=\frac{1+\theta}{4\theta}$: retailers do not incur the cost of excess inventory in the low-demand states with a buyback contract, but they do with resale price maintenance.
\end{note}







\newpage
\section{Coordinating the newsvendor with demand updating}

\subsection{Model and analysis}
\begin{note}
    Donohue 2020.
\end{note}
Let $\xi\geq 0$ be the realization of that demand signal. Let $G(\cdot)$ be its distribution function and $g(\cdot)$ its density function. Let $F(\cdot\mid\xi)$ be the distribution given signal and it's stochastically increasing in $\xi$. Let period $1$ be the time before the demand signal and $2$ the time between the demand signal and the start of the selling season. 

Let $q_i$ be the retailer's total order as of period $i$\footnote{$q_1$ and $q_2-q_1$}. Let $c_i$ be the supplier's per unit production cost in period $i$, with $c_1<c_2$. The supplier charges $w_i$ in period $i$. Also, the supplier offers buyback for $b$ per unit unsold. Let $p$ be the retail price. Normalize to zero the salvage value of leftover inventory and any indirect costs due to lost sales. No holding cost on inventory carried from period 1 to period 2.

Begin with period $2$. Let $\Omega_2(q_2\mid q_1,\xi)$ be the supply chain's expected revenue minus the period 2 production cost:
\begin{equation}
    \Omega(q_2\mid q_1,\xi)=p S(q_2\mid \xi)-c_2 q_2+c_2 q_1.
\end{equation}
Let $q_2(q_1,\xi)$ be the supply chain's optimal $q_2$ given $q_1$ and $\xi$. Let $q_2(\xi)=q_2(0,\xi)$, i.e., $q_2(\xi)$ is the optimal order if the retailer has no inventory at the start of period 2. Given $\Omega_2(q_2\mid q_1,\xi)$ is strictly concave in $q_2$,
\begin{equation}
    F(q_2(\xi)\mid\xi)=\frac{p-c_2}{p}.
\end{equation}
$q_2(\xi)$ is increasing in $\xi$\footnote{$F(\cdot\mid \xi)$ is SI in $\xi$ and RHS is unchanged in $\xi$.}, so it is possible to define the function $\xi(q_1)$ such that 
\begin{equation}
    F(q_1\mid\xi(q_1))=\frac{p-c_2}{p}.
\end{equation}

We have the retailer's period 2 expected profit
\begin{equation*}
    \pi_2(q_2\mid q_1,\xi)=(p-b)S(q_2\mid \xi)-(w_2-b)q_2+w_2 q_1,
\end{equation*}
where assume the supplier delivers the retailer's order in full. Choose $\lambda\in[0,1]$ and 
\begin{align*}
    p-b&=\lambda p\\
    w_2-b&=\lambda c_2.
\end{align*}
With this contract we have 
\begin{equation*}
    \pi_2(q_2\mid q_1,\xi)=\lambda(\Omega_2(q_2\mid q_1,\xi)-c_2 q_1)+w_2 q_1.
\end{equation*}
Thus, $q_2(q_1,\xi)$ is also the retailer's optimal order, i.e., the contract coordinates the retailer's \textbf{period 2 decision.}

Now consider whether the \textbf{supplier} indeed fills the retailer's entire period 2 order. Let $x$ be the total inventory in the supply chain at the start of period 2 with $x\geq q_1$. Let $y$ be the inventory at the retailer after the supplier's delivery in period 2. Let $\Pi_2(y\mid x, q_1,\xi)$ be the supplier's profit, where $x\leq y\leq q_2$,
\begin{align*}
    \Pi_2(y\mid x,q_1,q_2,\xi)&=b S(y\mid\xi)-by+w_2(y-q_1)-(y-x)c_2\\
    &=(1-\lambda)(\Omega_2(y\mid q_1,\xi)-c_2 q_1)+c_2 x- w_2 q_1
\end{align*}
where the above follows from the contract terms, $w_2=\lambda c_2+b$. Given $q_2>x$, the supplier fills the order as long as $q_2\leq q_2(q_1,\xi)$. 

In \textbf{period 2}, assuming a coordinating $\{w_2,b\}$ pair is chosen, the retailer's expected profit is\footnote{$-w_1 q_1+E\left[\pi_2(q_2\mid q_1,\xi)\right]$} 
\begin{equation*}
    \pi_1(q_1)=-(w_1-w_2+\lambda c_2)q_1+\lambda E\left[\Omega_2(q_2(q_1,\xi)\mid q_1,\xi)\right].
\end{equation*}
The supply chain's expected profit is 
\begin{equation*}
    \Omega_1(q_1)=-c_1 q_1+E\left[\Omega_2(q_2(q_1,\xi)\mid q_1,\xi)\right].
\end{equation*}
Choose $w_1$ so that 
\begin{equation*}
    w_1-w_2+\lambda c_2=\lambda c_1
\end{equation*}
because then
\begin{equation*}
    \pi_1(q_1)=\lambda \Omega_1(q_1),
\end{equation*}
i.e., the supply chain coordinates.
Given $\Omega_1(q_1)$ is strictly concave, $q_1^o$ follows:
\begin{align}
     \frac{\partial\Omega_1(q_1^o)}{\partial q_1}&=-c_1+c_2(1-G(\xi(q_1^o)))+\int_0^{\xi(q_1^o)} p S^\prime(q_1^o\mid\xi) g(\xi)d \xi \nonumber\\
    &=0 \label{eq:6.4}
\end{align}


Assuming the supplier fills the retailer's period 2 order, the supplier's period 2 profit is 
$$
\begin{aligned}
\Pi_{2}\left(x, q_{1}, q_{2}, \xi\right) &=b S\left(q_{2} \mid \xi\right)-b q_{2}-\left(q_{2}-x\right)^{+} c_{2} \\
&=(1-\lambda) \Omega_{2}\left(q_{2} \mid q_{1}, \xi\right)-w_{2} q_{2}+x c_{2}-\left(x-q_{2}\right)^{+} c_{2}
\end{aligned}
$$
Given that $q_{2} \geq q_{1}$, the above is strictly increasing in $x$ for $x \leq q_{1}$. Hence, the supplier surely produces and delivers the retailer's period 1 order (as long as $\left.q_{1} \leq q_{1}^{o}\right)$. The supplier's period 1 expected profit is
$$
\begin{aligned}
\Pi_{1}\left(x \mid q_{1}\right)=&-c_{1} x+E\left[\Pi_{2}\left(x, q_{1}, q_{2}, \xi\right)\right] \\
=&-c_{1} x+E\left[(1-\lambda) \Omega_{2}\left(q_{2} \mid q_{1}, \xi\right)\right]-w_{2} q_{2}+x c_{2} \\
&-c_{2} \int_{0}^{\xi(x)}\left(x-q_{2}(\xi)\right) g(\xi) \mathrm{d} \xi
\end{aligned}
$$
It follows that
$$
\frac{\partial \Pi_{1}\left(x \mid q_{1}\right)}{\partial x}=-c_{1}+c_{2}(1-G(\xi(x)))
$$
and from \autoref{eq:6.4}
$$
\frac{\partial \Pi_{1}\left(q_{1}^{\mathrm{o}} \mid q_{1}^{\mathrm{o}}\right)}{\partial x}=-c_{1}+c_{2}\left(1-G\left(\xi\left(q_{1}^{\mathrm{o}}\right)\right)\right)<0 ,
$$
i.e., the supplier has no incentive to produce more than $q_1^o$ given the retailer orders $q_1^o$. Hence, with a coordinating $\{w_1,w_2,b\}$ contract the supplier produces just enough inventory to cover the retailer's period 1 order. 
\begin{note}
    $$w_2-c_2=w_1-(\lambda c_1+(1-\lambda)c_2)<w_1-c_1,$$
    i.e. with a coordinating contract the supplier's margin in period 2 is actually \textbf{lower} than in period 1, which contrasts with intuition that the supplier should charge a higher margin for the later production since it offers the retailer an additional benefit over early production.
\end{note}

% \subsection{Discussion}





\newpage
\section{Coordination in the single-location base-stock model}
This section considers a model with perpetual demand and many replenishment opportunities. Hence, the newsvendor model is not appropriate. 

\subsection{Model and analysis}
Suppose a supplier sells a single product to a single retailer. Let $L_r$ be the lead time to replenish an order from the retailer. The supplier has infinite capacity so the supplier keeps no inventory and the retailer's replenishment lead time is always $L_r$. Let $\mu_r=E[D_r]$ and $F_r,f_r$ of $D_r$ with $F_r$ strictly increasing and $F_r(0)=0$, which rules out the possibility that it is optimal to carry no inventory. 

The retailer incurs inventory holding costs at rate $h_r>0$ per unit of inventory. $\beta_r,\beta_s$ the backorder penalty. 

Let $I_r(y)$ be the retailer's expected inventory at time $t+L_r$ when the retailer's inventory level is $y$ at time $t$:
\begin{equation}
    I_r(y)=\int_0^y (y-x)f_r(x)dx=\int_0^y F_r(x)dx.
\end{equation}
Let $B_r(y)$ be the expected backorders
\begin{equation}
    B_r(y)=\int_y^\infty (x-y)f_r(x)dx=\mu_r-y+I_r(y).
\end{equation}
With a base-stock policy, the retailer continuously orders inventory and chooses $s_r$.

Let $c_r(s_r)$ be the retailer's average cost per unit time when the retailer implements the base-stock policy $s_r$:
\begin{align*}
    c_r(s_r)&=h_r I_r(s_r)+\beta_r B_r(s_r)\\
    &=\beta_r(\mu_r-s_r)+(h_r+\beta_r)I_r(s_r).
\end{align*}
The supplier's expected cost function is 
\begin{align*}
    c_s(s_r)&=\beta_s B_r(s_r)\\
    &=\beta_s(\mu_r-s_r+I_r(s_r)).
\end{align*}
Let $c(s_r)$ be the supply chain's expected cost per unit time,
\begin{align}
    c(s_r)&=c_r(s_r)+c_s(s_r)\nonumber\\
    &=\beta(\mu_r-s_r)+(h_r+\beta)I_r(s_r).\label{eq:7.3}
\end{align}
$c\left(s_{\mathrm{r}}\right)$ is strictly convex, so there is a unique supply chain optimal base-stock level, $s_{\mathrm{r}}^{\mathrm{o}}$. It satisfies the following critical ratio equation
$$
I_{\mathrm{r}}^{\prime}\left(s_{\mathrm{r}}^{\mathrm{o}}\right)=F_{\mathrm{r}}\left(s_{\mathrm{r}}^{\mathrm{o}}\right)=\frac{\beta}{h_{\mathrm{r}}+\beta}
$$
Let $s_{\mathrm{r}}^{*}$ be the retailer's optimal base-stock level. The retailer's cost function is also strictly convex, so $s_{\mathrm{r}}^{*}$ satisfies
$$
F_{\mathrm{r}}\left(s_{\mathrm{r}}^{*}\right)=\frac{\beta_{\mathrm{r}}}{h_{\mathrm{r}}+\beta_{\mathrm{r}}} .
$$
Given 	$\beta_r<\beta$, it follows from the above two expressions that $s_r^* < s_r^o$, i.e., the retailer chooses a base-stock level that is \textbf{less than optimal}.

Suppose the supplier agree to transfer at every time $t$
$$t_I I_r(y)+t_B B_r(y)$$
where $y$ is the retailer's inventory level at time $t$ and $t_I$ and $t_I$ are constants. Further more, consider $\lambda\in(0,1]$\footnote{If $\lambda=0$, then any base-stock level is optimal},
\begin{align*}
    t_I&=(1-\lambda)h_r\\
    t_B&=\beta_r-\lambda\beta
\end{align*}
The retailer's expected cost function is now 
\begin{equation}\label{eq:7.5}
    c_r(s_r)=(\beta_r-t_B)(\mu_r-s_r)+(h_r+\beta_r-t_I-t_B)I_r(s_r).
\end{equation}
It follows from \autoref{eq:7.3} and \autoref{eq:7.5} that
\begin{equation}
    c_r(s_r)=\lambda c(s_r)\label{eq:7.6}
\end{equation}
Hence, $s_r^o$ minimizes the retailer's cost and the contracts coordinate the supply chain. 





\newpage
\section{Coordination in the two-location base-stock model}
\subsection{Model}
Let $h_s$, $0<h_s<h_r$, be the supplier's per unit holding cost rate incurred with on-hand inventory. Let $D_s<0$ be demand during an interval of time with length $L_s$. 

Both firms use base-stock policies to manage inventory.

\subsection{Cost function}
Let $c_i(s_r,s_s)$ be the average rate at which firm $i$ incurs costs at the retail level. 
At time $t+L_s$, the retailer's inventory level is $s_r-\left(D_s-s_s\right)^+$. So
\begin{equation*}
    c_i(s_r,s_s)=F_s(s_s)c_i(s_r)+\int_{s_s}^\infty c_i(s_r+s_s-x)f_s(x)dx.
\end{equation*}
Let $I_r(s_r,s_s)$ and $B_r(s_r,s_s)$ be the retailer's average inventory and backorders given the base-stock levels:
\begin{align*}
    I_r(s_r,s_s)&=F_s(s_s)I_r(s_r)+\int_{s_s}^\infty I_r(s_r+s_s-x)f_s(x)dx\\
    B_r(s_r,s_s)&=F_s(s_s)B_r(s_r)+\int_{s_s}^\infty B_r(s_r+s_s-x)f_s(x)dx\\
\end{align*}
Let $\pi_i(s_r,s_s)$ be firm $i$'s total average cost rate. We have 
$$\pi_r(s_r,s_s)=c_r(s_r,s_s).$$
Let $I_s(s_s)$ be the supplier's average inventory. Analogous to the retailer's functions, we have 
$$I_s(y)=\int_0^y F_s(x)dx.$$
The supplier's average cost is 
$$\pi_s(s_r,s_s)=h_s I_s(s_s)+c_s(s_r,s_s).$$
The supply chain's total cost is $\Pi(s_r,s_s)=\pi_r(s_r,s_s)+\pi_s(s_r,s_s)$.

\subsection{Behavior in the decentralized game}
Let $s_i(s_j)$ be an optimal base-stock level for firm $i$ given firm $j$'s strategy. 
A Nash equilibrium $\{s_r^*,s_s^*\}$ is $s_r^*=s_r (s_s^*)$ and $s_s^*=s_s(s_r^*)$. It's unique and exists. But the penalty is positive. The uniqueness requires
\begin{equation}
    |s_i^\prime(s_j)|<1.
\end{equation}

\subsection{Coordiantion with linear transfer payments}
Supply chain coordination in this setting is achieved when $\{s_r^o,s_s^o\}$ is a Nash equilibrium. 

Suppose the supplier offers
\begin{equation*}
    t_I I_r(s_r,s_s)+t_B^r B_r(s_r,s_s)+t_B^s B_s(s_s),
\end{equation*}
where $t_I, t_B^r$ and $t_B^s$ are constants and $B_s(s_s)$ is the supplier's average backorder:
\begin{equation*}
    B_s(y)=\mu_s-y+I_s(y).
\end{equation*}

Given $\Pi\left(s_{\mathrm{r}}, s_{\mathrm{s}}\right)$ is continuous, any optimal policy with $s_{\mathrm{s}}>0$ must set the following two marginals to zero
\begin{equation}
    \frac{\partial \Pi\left(s_{\mathrm{r}}, s_{\mathrm{s}}\right)}{\partial s_{\mathrm{r}}}=F_{\mathrm{s}}\left(s_{\mathrm{s}}\right) c^{\prime}\left(s_{\mathrm{r}}\right)+\int_{s_{\mathrm{s}}}^{\infty} c^{\prime}\left(s_{\mathrm{r}}+s_{\mathrm{s}}-x\right) f_{\mathrm{s}}(x) \mathrm{d} x
\end{equation}
and
\begin{equation}
    \frac{\partial \Pi\left(s_{\mathrm{r}}, s_{\mathrm{s}}\right)}{\partial s_{\mathrm{s}}}=F_{\mathrm{s}}\left(s_{\mathrm{s}}\right) h_{\mathrm{s}}+\int_{s_{\mathrm{s}}}^{\infty} c^{\prime}\left(s_{\mathrm{r}}+s_{\mathrm{s}}-x\right) f_{\mathrm{s}}(x) \mathrm{d} x .
\end{equation}
Since $F_{\mathrm{s}}\left(s_{\mathrm{s}}\right)>0$, there is only one possible optimal policy with $s_{\mathrm{s}}>0,\left\{\tilde{s}_{\mathrm{r}}^{1}, \tilde{\mathrm{s}}_{\mathrm{s}}^{1}\right\}$, where $\tilde{s}_{\mathrm{r}}^{1}$ satisfies
\begin{equation}\label{eq:8.4}
    c^{\prime}\left(\tilde{s}_{\mathrm{r}}^{1}\right)=h_{\mathrm{s}},
\end{equation}
and $\tilde{s}_s^1$ satisfies $\partial\Pi(\tilde{s}_r^1,\tilde{s}_s^1)/\partial s_s=0$. \autoref{eq:8.4} simplifies 
\begin{equation*}
    F_r(\tilde{s}_r^1)=\frac{h_s+\beta}{h_r+\beta},
\end{equation*}
so $\tilde{s}_r^1$ exists and is unique. Also, $\Pi(\tilde{s}_r^1,s_s)$ is strictly convex in $s_s$. $\tilde{s}_s^1$ also exists and is unique.

\begin{note}
    There are too many questions. TBD.
\end{note}








\newpage
\section{Coordination with internal markets}
\subsection{Model and analysis}
Suppose there is one supplier, one production manager and two retailers. The production manager is the supplier's employee and the retailers are independent firms.

\begin{itemize}
    \item The production manager chooses a product input level $e$, which yields an output of $Q=Y e$ finished units, where $Y\in\left[0,1\right]$ is a random variable.
    \item The production manager incurs cost $c(e)$, where $c(e)$ is strictly convex and increasing.
    \item Retailer $i$ observes $\alpha_i$, the realization of $A_i>0$. Each retailer submits an order to the supplier.
    \item The supplier allocates $q_i$ to retailer $i$ with $q_i$ not exceeding the order and $q_1+q_2\leq Q$.
    \item Retailer $i
    $ earns $q_i p_i(q_i)$ where $p_i(q_i)=\alpha_i q_i^{-1/\eta}$ and $\eta>1$ is the constant demand elasticity. 
    \item Let $\theta$ be the realization of $Y$, $A=\left\{A_1,A_2\right\} $and $\alpha=\left\{\alpha_1,\alpha_2\right\}$.
\end{itemize}

\textbf{First consider the supply chain optimal actions}. 
Let $\gamma$ be the fraction of $Q$ that is allocated to retailer $1$. The total retailer revenue is 
\begin{equation*}
    \pi(\gamma,\alpha,Q)=\left(\alpha_1 \gamma^{(\eta-1)/\eta}+\alpha_2 (1-\gamma)^{(\eta-1)/\eta}\right)Q^{(\eta-1)/\eta}.
\end{equation*}
Let $\gamma^{\circ}(\alpha)$ be the optimal share to allocate to retailer one:
\begin{equation}
    \gamma^{o}(\alpha)=\frac{\alpha_{1}^{\eta}}{\alpha_{1}^{\eta}+\alpha_{2}^{\eta}}
\end{equation}
Conditional on an optimal allocation, the retailers' total revenue is
$$
\pi(\alpha, Q)=\pi\left(\gamma^{\circ}(\alpha), \alpha, Q\right)=\left(\alpha_{1}^{\eta}+\alpha_{2}^{\eta}\right)^{1 / \eta} Q^{(\eta-1) / \eta} .
$$
Total expected supply chain profit, $\Pi(e, A, Y)$, is thus
$$
\Pi(e, A, Y)=E[\pi(A, Y e)]-c(e) .
$$
Profit is concave in $e$, so the unique optimal production effort level, $e^{\mathrm{o}}$, satisfies
\begin{equation}\label{eq:9.2}
    \Pi^{\prime}\left(e^{\mathrm{o}}\right)=\left(\frac{\eta-1}{\eta}\right)\left(e^{\mathrm{o}}\right)^{-1 / \eta} E\left[\left(A_{1}^{\eta}+A_{2}^{\eta}\right)^{1 / \eta} Y^{(\eta-1) / \eta}\right]-c^{\prime}\left(e^{\mathrm{o}}\right)=0.
\end{equation}

Now consider \textbf{decentralized operations}. 
Suppose the supplier charges the reatilers $w$ per unit. Then retailer $i$'s profit is 
\begin{equation*}
    \pi_i(q_i,w)=\alpha_i q_i^{(\eta-1)\eta}-w q_i.
\end{equation*}
Retailer $i$'s optimal quantity, $q_{i}^{*}$, satisfies
$$
\frac{\partial \pi_{i}\left(q_{i}^{*}, w\right)}{\partial q_{i}}=0=\left(\frac{\eta-1}{\eta}\right) \alpha_{i}\left(q_{i}^{*}\right)^{-1 / \eta}-w .
$$
It follows that $q_{i}^{*}=\gamma^{\circ}(\alpha) Q$ when $w=w(\alpha, Q)$,
$$
w(\alpha, Q)=\left(\frac{\eta-1}{\eta}\right)\left(\alpha_{1}^{\eta}+\alpha_{2}^{\eta}\right)^{1 / \eta} Q^{-1 / \eta}
$$
\begin{note}
    $$\frac{\partial\pi(\alpha,Q)}{\partial Q}=w(\alpha,Q).$$
\end{note}

The supplier could offer the retailers a contract that identifies $w(\alpha,Q)$ as the wholesale price contingent on the realization of $A$ and $Y$. The supplier merely commits at the start of the game to hold a market for output after the retailers observe their demand realizations. The unique market-clearing price is $w(\alpha,Q)$, and so the market optimizes the supply chain's profit without the supplier observing $A$.

Consider the\textbf{ production manager}. Assume the supplier can't observe $e$. But she can observe the final output $Ye$. So suppose the supplier pays the production manager
\begin{equation}\label{eq:9.3}
    \begin{aligned}
        &\left(\frac{\eta-1}{\eta}\right)\left(e^{0}\right)^{-1 / \eta} E\left[\left(A_{1}^{\eta}+A_{2}^{\eta}\right)^{1 / \eta} Y^{(\eta-1) / \eta}\right] / E[Y]\\
        =&E\left[Q w(A, Q) \mid e^{0}\right] / E\left[Q \mid e^{\mathrm{o}}\right]
    \end{aligned}
\end{equation}
per unit of realized output. 
Hence, the supplier pays the production manager the \textbf{expected shadow price} of capacity and sells capacity to the retailer for the \textbf{realized shadow price} of capacity. The production manager's expected utility is 
$$
u(e)=\left(\frac{\eta-1}{\eta}\right)\left(e^{\mathrm{o}}\right)^{-1 / \eta} E\left[\left(A_{1}^{\eta}+A_{2}^{\eta}\right)^{1 / \eta} Y^{(\eta-1) / \eta}\right] e-c(e)
$$
and the marginal utility is
$$
u^{\prime}(e)=\left(\frac{\eta-1}{\eta}\right)\left(e^{\mathrm{o}}\right)^{-1 / \eta} E\left[\left(A_{1}^{\eta}+A_{2}^{\eta}\right)^{1 / \eta} Y^{(\eta-1) / \eta}\right]-c^{\prime}(e) .
$$
From \autoref{eq:9.2} we know the production manager's optimal effort is $e^o$. 
\begin{note}
    The supplier earns zero profit in expectation from the internal market. $E[Qw(A,Q)|e^o]$ is the expected revenue from the retailers, and from \autoref{eq:9.3}, it's also the expected payout to the production manager.
    Kouvelis and Lariviere (2000) 
\end{note}







% \nocite{*}
% \printbibliography
\end{document}