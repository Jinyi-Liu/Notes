\documentclass[12pt]{article}
\usepackage[utf8]{inputenc}	% Para caracteres en español
\usepackage{amsmath,amsthm,amsfonts,amssymb,amscd}
\usepackage{multirow,booktabs}
\usepackage[table]{xcolor}
\usepackage{fullpage}
\usepackage{lastpage}
\usepackage{enumitem}
\usepackage{fancyhdr}
\usepackage{mathrsfs}
\usepackage{wrapfig}
\usepackage{setspace}
\usepackage{calc}
\usepackage{multicol}
\usepackage{cancel}
\usepackage[retainorgcmds]{IEEEtrantools}
\usepackage[margin=4cm]{geometry}
\usepackage{amsmath}
\newlength{\tabcont}
\setlength{\parindent}{0.0in}
\setlength{\parskip}{0.00in}
\usepackage{empheq}
\usepackage{framed}
\usepackage[most]{tcolorbox}
\usepackage{xcolor}
\usepackage{hyperref}

% \usepackage[style=alphabetic,sorting=nyt,sortcites=true,autopunct=true,babel=hyphen,hyperref=true,abbreviate=false,backref=true,backend=biber]{biblatex}
% \addbibresource{bibliography.bib}

\colorlet{shadecolor}{orange!15}
\parindent 0in
\parskip 12pt
\geometry{margin=1in, headsep=0.25in}
\theoremstyle{definition}
\newtheorem{defn}{Definition}
\newtheorem{reg}{Rule}
\newtheorem{exer}{Exercise}
\newtheorem{note}{Note}

% \usepackage{amsmath}
\DeclareMathOperator*{\argmax}{arg\,max}
\DeclareMathOperator*{\argmin}{arg\,min}
\newcommand{\oF}{\overline{F}(q)}

\begin{document}
\setcounter{section}{1}
\numberwithin{equation}{section}

\title{Chapter 2 Review Notes}

\thispagestyle{empty}

\begin{center}
{\LARGE \bf Notes}\\
% {\large }\\
\end{center}


\section{Coordinating the newsvendor}

With the standard wholesale-price contract, it is shown that the retailer does not order enough inventory to maximize the supply chain’s total profit because the retailer ignores the impact of his action on the supplier’s profit. Hence, coordination requires that the retailer be given an **incentive** to increase his order

Types of contracts to coordinate the supply chain and arbitrarily divide its profit:
- Buyback contracts
- revenue-sharing contracts
- quantity-flexibility contracts 
- sales-rebate contracts
- quantity-discount contracts


\subsection{Model and analysis}
$\mu=E[D]$ is the mean of demand. The supplier's production cost per unit is $c_s$ and the retailer's marginal cost per unit is $c_r$, $c_s+c_r<p$. $c_r$ is incurred upon procuring a unit. Goodwill penalty cost $g_r$ and the analogous cost for the supplier is $g_s$. Let $c=c_s+c_r$ and $g=g_s+g_r$. $v$ is net of any salvage expenses. 

The details of the negotiation process is not explored.

Each firm is risk neutral. Full information.

- voluntary compliance
- forced compliance
- The approach taken in this section is to assume forced compliance but to check if the supplier has an incentive to deviate from the proposed contractual terms.


\begin{align*}
    S(q)&=E[\min(q,D)]=q(1-F(q))+\int_0^q y f(y)dy=q-\int_0^q F(y)dy\\
    I(q)&=E[(q-D)]^+=q-S(q)\\
    L(q)&=E[(D-q)^+]=\mu-S(q)
\end{align*}

where $I(q)$ is the expected leftover inventory and $L(q)$ is the lost-sales function.

The retailer's profit function is 
\begin{equation}\label{eq:retailer}\tag{Retailer}
    \begin{aligned}
        \pi_r(q)&=pS(q)+v I(q)-g_r L(q)-c_r q - T\\
    &=(p-v+g_r)S(q)-(c_r-v)q-g_r\mu-T,
    \end{aligned}
\end{equation}
the supplier's profit function is 
\begin{equation}\label{eq:supplier}\tag{Supplier}
    \pi_s(q)=g_s S(q)-c_s q-g_s\mu + T,
\end{equation}
and the supply chain's profit function is 
\begin{equation}\label{eq:2.1}
    \Pi(q)=\pi_r(q)+\pi_s(q)=(p-v+g)S(q)-(c-v)q-g\mu
\end{equation}

Let $q^o$ be a supply chain optimal order quantity, we have 
\begin{equation}
    S^\prime(q^o)=\overline F(q^o)=\frac{c-v}{p-v+g}
\end{equation}

since $F$ is strictly increasing and thus $\Pi$ is strictly concave and the optimal order quantity is unique.

Let $q_r^*=\argmax \pi_r(q)$

\subsection{The wholesale-price contract}\
Let $T_w(q,w)=w q$. Since $\pi_r(q,w)$ is strictly concave in $q$, we have 

\begin{equation}\label{eq:2.3}
    (p-v+g_r)S^\prime(q_r^*)-(w+c_r-v)=0.
\end{equation}

Since $S^\prime(q)$ is decreasing, $q_r^*=q^o$ only when
$$
w=(\frac{p-v+g_r}{p-v+g})(c-v)-(c_r-v).
$$
It shows that $w\leq c_s$, i.e., coordinates only if the supplier earns a \textbf{nonpositive} profit. Thus the wholesale-price contract is generally \textbf{not considered} a coordinating contract.

From \autoref{eq:2.3} we have 
$$F(q_r^*)=1-\frac{w+c_r-v}{p-v+g_r}$$
It's obvious that there is a one-for-one mapping between $w$ and $q_r^*$, then we have $$w(q)=(p-v+g_r)\overline{F}(q)-(c_r-v),$$
the unique wholesale price that induces the retailer to order $q_r^*$ units.
Then we have the supplier's profit function:
\begin{equation}
    \pi_s(q,w(q))=g_s S(q)+(w(q)-c_s)q-g_s\mu\label{eq:2.4},
\end{equation}
from this we know that the \textit{compliance regime} \textbf{does not} matter with this contract: for a fixed $w$ no less than $c_s$ the supplier's profit  is nondecreasing in $q$.

We have the supplier's marginal profit:
\begin{align*}
    \frac{\partial\pi_s(q,w(q))}{\partial q}&=g_s S^\prime(q)+w(q)-c_s+w^\prime(q)q\\
    &=(p-v+g_r)\overline{F}(q)\left(1+\frac{g_s}{p-v+g_r}-\frac{q f(q)}{\overline{F}(q)}\right)-(c-v)
\end{align*}
$\pi_s(q,w(q))$ is decreasing in $q$ if $qf(q)/\overline{F}(q)$ is increasing. This type of demand distributions are called increasing generalized failure rate (\textbf{IGFR}) distributions.

Similarly, from \ref{eq:retailer} we have 
\begin{align*}
    \pi_r(q,w(q))&=(p-v+g_r)S(q)-(c_r-v)q-g_rv-w(q)q\\
    &=(p-v+g_r)\left(S(q)-\oF q\right)-g_rv
\end{align*}
then we have 
\begin{equation*}
    \frac{\partial\pi_r(q,w(q))}{\partial q}=(p-v+g_r)f(q)q>0,
\end{equation*}
so the supplier can increase the retailer's profit by reducing the price. 
The supply chain's profit is increasing in $q$ for $[q_s^*,q^o]$ and so is the retailer's profit. Hence, \textbf{an increase in retail power can actually improve supply chain performance.} 

Define the efficiency of the contract, $\Pi(q_s^*)/\Pi(q^o)$ and $\pi_s(q_s^*,w(q_s^*))/\Pi(q_s^*)$, the supplier's profit share. For a broad set of demand distributions, the argument that the retailer is being compensated for \textbf{the risk that demand and supply do no match} holds, where both measures approach 1 with the variation approach 0.\cite{lariviere_selling_2001}

Two-period version of the model which has excess inventory and demand updating. \textit{Push} and \textit{pull} strategies. Advanced purchase discount $w_1<w_2$. The supply chain effciency is substantially higher. There exist conditions in which advanced purchase discounts coordinate the supply chain and arbitrarily allocate its profit. \textbf{TBD}

\subsection{The buyback contract}
With a buyback contract the supplier charges the retailer $w$ per unit puchased, but pays the retailer $b$ per unit remaining at the end of the season:
$$T_b(q,w,b)=wq-bI(q)=b S(q)+(w-b)q.$$
See \cite{pasternack_optimal_1985} for detail. An important \textbf{implicit} assumption is that the supplier is able to verify the number of remaining units and the cost of such monitoring does not negate the benefits created by the contract.

The retailer's profit now is:
$$\pi_{\mathrm{r}}\left(q, w_{\mathrm{b}}, b\right)=\left(p-v+g_{\mathrm{r}}-b\right) S(q)-\left(w_{\mathrm{b}}-b+c_{\mathrm{r}}-v\right) q-g_{\mathrm{r}} \mu$$
Consider $\{w_b,b\}$ such that for $\lambda\geq 0$,
\begin{align}
    &p-v+g_r-b=\lambda(p-v+g)\\
    &w_b-b+c_r-v=\lambda(c-v)
\end{align}
A Comparing with \autoref{eq:2.1} leads to:
\begin{align}
    \pi_r(q,w_b,b)&=\lambda(p-v+g)S(q)-\lambda(c-v)q-g_r\mu\nonumber\\
    &=\lambda\Pi(q)+\mu(\lambda g-g_r).\label{eq:2.7}
\end{align}
The supplier's profit function is 
\begin{equation*}
    \pi_s(q,w_b,b)=(1-\lambda)\Pi(q)-\mu(\lambda g-g_r).
\end{equation*}
So the buyback contract \textbf{coordinates} with voluntary compliance as long as $\lambda\leq 1$. When $\lambda=1$ (or $\lambda=0$), the $q^o$ is optimal for the supplier (or retailer), but so is every other quantity since the profit function is not related with $q$. Hence, coordination is possible but no longer the uniuqe Nash equilibrium.

The $\lambda$ parameter acts to allocate the supply chain's profit between the two firms. The retailer earns the entire supply chain profit $\pi_r(q^o,w_b,b)=\Pi(q^o)$ when 
\begin{equation}\label{eq:2.8}
    \lambda=\frac{\Pi(q^o)+\mu g_r}{\Pi(q^o)+\mu g}\leq 1
\end{equation}
and the supplier $\pi_s(q^o,w_b,b)=\Pi(q^o)$, when
\begin{equation}\label{eq:2.9}
    0\leq \lambda=\frac{\mu g_r}{\Pi(q^o)+\mu g}.
\end{equation}
So \textbf{every} possible profit allocation is feasible with this set of coordinating contracts, assuming $\lambda=0$ and $\lambda=1$ are considered feasible.

\begin{note}
    The coordination of the supply chain requires the \textbf{simultaneous adjustment} of both the wholesale price $w_b$ and the buyback rate $b$. This has implications for the bargaining process, e.g., never negotiate those parameters sequentially.
\end{note}

\begin{note}
    Stock rebalancing in centralized system and decentralized system.
\end{note}

\subsection{The revenue-sharing contract}
With a revenue-sharing contract the supplier charges $w_r$ per unit purchased plus the retailer gives the supplier a percentage of his revenue. Let $\phi$ be the fraction of revenue that retailer keeps.

The transfer payment with revenue sharing is 
\begin{align*}
    T_r(q,w_r,\phi)&=w_r q+(1-\phi)(vI(q)+pS(q))\\
    &=(w_r+(1-\phi)v)q+(1-\phi)(p-v)S(q)
\end{align*}
The retailer's profit function is
$$
\pi_{\mathrm{r}}\left(q, w_{\mathrm{r}}, \phi\right)=\left(\phi(p-v)+g_{\mathrm{r}}\right) S(q)-\left(w_{\mathrm{r}}+c_{\mathrm{r}}-\phi v\right) q-g_{\mathrm{r}} \mu
$$
Now consider the set of revenue-sharing contracts, $\left\{w_{\mathrm{r}}, \phi\right\}$, such that $\lambda \geq 0$ and
$$
\begin{aligned}
&\phi(p-v)+g_{\mathrm{r}}=\lambda(p-v+g) \\
&w_{\mathrm{r}}+c_{\mathrm{r}}-\phi v=\lambda(c-v)
\end{aligned}
$$
Now we have 
\begin{align}
    &\pi_r(q,w_r,\phi)=\lambda\Pi(q)+\mu(\lambda g - g_r)\label{eq:2.10}\\
    &\pi_s(q,w_r,\phi)=(1-\lambda)\Pi(q)-\mu(\lambda g-g_r)\nonumber.
\end{align}
It's obvious that \autoref{eq:2.8} and \autoref{eq:2.9} provides the same $\lambda$.

From \autoref{eq:2.10} and \autoref{eq:2.7} we find similarity. Consider a coordinating buyback contract $\{w_b,b\}$. The retailer pays $w-b-b$ for each unit purchased and an additional $b$ per unit sold. With revenue sharing the retailer pays $w_r+(1-\phi)v$ and $(1-\phi)(p-v)$.  Now they are equivalent when 
\begin{align*}
    w_b-b&=w_r+(1-\phi)v\\
    b&=(1-\phi)(p-v)
\end{align*}
\begin{note}
    Their path will diverge in more complex settings.
\end{note}




\numberwithin{equation}{section}
\newcommand{\spq}{S(q,p^o(q))}
\newcommand{\pq}{p^o(q)}
\thispagestyle{empty}
\section{Coordinating the newsvendor with \textit{price-dependent} demand}


\subsection{Model and analysis}
Now the retailer chooses his price in addition to his order quantity. Let $F(q|p)$ be the distribution function of demand, where $p$ is the retail price. Assume $\frac{\partial F(q|p)}{\partial p}>0$. To obtain initial insights, assume the retailer sets his \textbf{price} \textit{at the same time as} his \textbf{stocking decision} and the price is \textbf{fixed} throughout the season.
\begin{note}
    van Mieghem and Dada (1999). A hybrid model. The retailer chooses $q$, then observes a demand signal and then chooses price.
\end{note}

The integrated channel's profit is 
\begin{equation*}
    \Pi(q,p)=(p-v+g)S(q,p)-(c-v)q-g\mu
\end{equation*}
where $S(q,p)$ is expected sales given the stocking quantity $q$ and the price $p$, and similarly, we have
\begin{equation*}
    S(q,p)=q-\int_0^q F(y|p)dy
\end{equation*}
\begin{note}
    The integrated channel profit function need not be concave nor unimodal (Petruzzi \& Dada 1999)
\end{note}
Let $\pq$ be the supply chain optimal price for a given $q$. The necessary condition for coordination is
\begin{equation}\label{eq:3.1}
    \frac{\partial\Pi(q,p^o(q))}{\partial p}=S(q,p^o(q))+(p^o(q)-v+g)\frac{\partial S(q,p^o(q))}{\partial p}=0.
\end{equation}
\begin{note}
    Either not satisfy the first-order condition or fail to coordinate the quantity decision.
\end{note}

Consider the \textbf{quantity-flexibility} contract. The retailer's profit function is 
\begin{align*}
    \pi_r(q,p,w_q,\delta)=&(p-v+g_r)S(q,p)-(w_q+c_r-v)q\\
    +&(w_q+c_r-v)\int_{(1-\delta)q}^q F(y|p)dy-\mu g_r
\end{align*}
For price coordination the first-order condition must hold,

\begin{align}
\frac{\partial \pi_{\mathrm{r}}\left(q, p^{\mathrm{o}}(q), w_{q}, \delta\right)}{\partial p}=& S\left(q, p^{\mathrm{o}}(q)\right)+\left(p^{\mathrm{o}}(q)-v+g_{\mathrm{r}}\right) \frac{\partial S\left(q, p^{\mathrm{o}}(q)\right)}{\partial p} \nonumber\\
&+\left(w_{q}+c_{\mathrm{r}}-v\right) \int_{(1-\delta) q}^{q} \frac{\partial F\left(y \mid p^{\mathrm{o}}(q)\right)}{\partial p} \mathrm{~d} y \nonumber\\
=& 0\label{eq:3.2}
\end{align}
The second term in \autoref{eq:3.2} is no smaller than the second term in \autoref{eq:3.1}\footnote{The assumption of $\partial F(q|p)/\partial p>0.$}, so the above holds only if the third term is nonpositive. 
But the third term is nonnegative as $w_q+c_r-v\geq 0$, so with a coordinating $w_q$, the coordination of price can only occur if $g_s=0$ and either $w_q=v-c_r$ or $\delta=0$. \textbf{Neither} is desirable. With $w_q=v-c_r$, then supplier has $w_q<c_s$\footnote{Why? An assumption?} which is not acceptable. With $\delta=0$ the contract degenerates to just a wholesale-price contract, so the retailer's quantity action is not optimal. Hence, the quantity-flexibility contract does not coordinate the newsvendor with price-dependent demand.

The \textbf{sales-rebate} contract does not fare better:
\begin{align*}
    \frac{\partial\pi_r(q,\pq,w_s,r,t)}{\partial p}=&\spq+(\pq-v+g_r)\frac{\partial\spq}{\partial p}\\
    &-r\int_t^q\frac{\partial F(y|\pq)}{\partial p}dy
\end{align*}
Since the last term is negative when $r>0$ and $t<q$, we know that the retailer prices below the optimal price\footnote{The above derivative is negative. Why it means that the retailer prices below the optimal price?}. Coordination might be achieved if there is something to induce the retailer to a higher price.

Now consider a \textbf{buyback} contract. The retailer's profit function is 
\begin{equation*}
    \pi_r(q,p,w_b,b)=(p-v+g_r-b)S(q,p)-(w_b-b+c_r-v)q-g_r\mu.
\end{equation*}
For coordination we must have the first-order condition:
\begin{equation}\label{eq:3.3}
    \frac{\partial\pi_r(q,\pq,w_b,r,t)}{\partial p}=\spq+(\pq-v+g_r-b)\frac{\partial\spq}{\partial p}=0.
\end{equation}
But comparing with \autoref{eq:3.1} it holds only if $b=-g_s< 0$ which violates that $b\geq 0$\footnote{If $g_s=0$, then $w_b=c_s$ and $b_s=0$ which means that the supplier earns no positive profit.}. Therefore, a buyback contract does not coordinate the newsvendor with price-dependent demand.

The buyback contract fails to coordinate in this setting because the parameters of the coordinating contracts depend on the price: from \autoref{eq:2.5} and \autoref{eq:2.6}, the coordinating parameters are
\begin{align*}
    b&=(1-\lambda)(p-v+g)-g_s\\
    w_b&=\lambda c_s+(1-\lambda)(p+g-c_r)-g_s.
\end{align*}
For a fixed $\lambda$, the coordianting buyback rate and wholesale price are linear in $p$. Hence, the buyback contract coordiantes the newsvendor with price-dependent demand if $b$ and $w_b$ are made \textbf{contingent} on the retail price chosen, or if $b$ and $w_b$ are chosen \textbf{after} the retailer commits to a price (but before the retailer chooses $q$). This is the \textbf{price-discount-sharing} contract\footnote{Bernstein and Federgruen (2000)}, which is called a "bill back" in practice. The retailer gets a lower wholesale price if the retailer reduces his price, i.e., the supplier shares in the cost of a price discount with the retailer. Then we have the retailer profit function:
\begin{align*}
    \pi_r(q,p,w_b,b)&=\lambda(p-v+g)S(q,p)-\lambda(c-v)q-g_r\mu\\
    &=\lambda(\Pi(q,p)+g\mu)-g_r\mu
\end{align*}
Hence, for the retailer as well ass the supplier, $\{q^o,p^o\}$ is optimal for $\lambda\in[0,1]$.

Now consider the \textbf{revenue-sharing} contract. The retailer's profit is 
\begin{equation*}
    \pi_r(q,p,w_r,\phi)=(\phi(p-v)+g_r)S(q,p)-(w_r+c_r-\phi v)q-g_r\mu.
\end{equation*}
Coordination require
\begin{equation}\label{eq:3.4}
    \frac{\pi_r(q,\pq,w_r,\phi)}{\partial p}=\spq+(\pq-v+g_r/\phi)\frac{\partial\spq}{\partial p}=0.
\end{equation}

\begin{itemize}
    \item Consider $g_r=g_s=0$. In this situation,
    \begin{equation*}
        \frac{\partial\pi_r(q,p,w_r,\phi)}{\partial p}=\frac{\partial\Pi(q,p)}{\partial p}
    \end{equation*}
    with \textbf{any} revenue-sharing contract. Thus, the retailer chooses $\pq$ no matter which revenue-sharing contract is chosen. Now revenue sharing is able to coordinate the retailer's quantity decision with precisely the same set of contracts used when the retailer prices is fixed. 

    Recall that with the \textit{fixed price} newsvendor \textbf{revenue sharing} and \textbf{buybacks} are equivalent. Here, the contracts produce different outcomes because with a buyback the retailer's share of revenue $(1-b/p)$ depends on the price, whereas with revenue sharing it is independent of the price, by definition\footnote{The above partial derivative}. However, the \textbf{price contingent buyback} contract (\textbf{price-discount} contract) is equivalent to revenue sharing: if $g_r=g_s=0$, the coordinating revenue-sharing contract yield 
    \begin{equation*}
        \pi_r(q,p,w_r,\phi)=\lambda\Pi(q,p)
    \end{equation*}
    from \autoref{eq:2.10}. And the price contingent buyback contract yield the same profit for any quantity and price from \autoref{eq:2.7},
    \begin{equation*}
        \pi_r(q,p,b(p),w_b(p))=\lambda\Pi(q,p).
    \end{equation*}
    \item Consider either $g_r>0$ or $g_s>0$.
\end{itemize}


























\thispagestyle{empty}
\section{Coordinating the newsvendor with effort-dependent demand}
\begin{note}
    Netessine and Rudi (2000a)
    Wang and Gerchak 2001
    Gilbert and Cvsa 2000
\end{note}
Only the quantity-discount contract can coordinate a retailer that chooses quantity, price and effort.

\subsection{Model and analysis}
\begin{itemize}
    \item Suppose a single effort level $e$, summarizes the retailer's activities and let $g(e)$ be the retailer's cost of exerting effort level $e$, where $g(0)=0$, $g^\prime(e)>0$ and $g^{\prime\prime}(e)>0$.
    \item  Assume there are no goodwill costs, $g_r=g_s=0$, $v=0$ and $c_r=0$. Let $F(q|e)$ be the distribution of demand given the effort level $e$, where demand is \textbf{stochastically increasing in effort}, i.e., $\partial F(q|e)/\partial e<0$.
    \item Suppose the retailer chooses his effort level \textbf{at the same time as} his order quantity. 
    \item Assume the supplier \textbf{cannot verify} the retailer's effort level, which implies the retailer cannot sign a contract binding the retailer to choose a particular effort level.
\end{itemize}

Then we have 
\begin{equation*}
    \Pi(q,e)=p S(q,e)-c q-g(e),
\end{equation*}
where 
\begin{equation*}
    S(q,e)=q-\int_0^q F(y|e)dy.
\end{equation*}
\textbf{The integrated channel’s profit function need not be concave nor unimodal.}
 Assume that the integrated channel solution is well behaved, i.e., $\Pi(q,e)$ is unimodal and maximized with finite arguments. $q^o$ and $e^o$ are the optimal solutions.

 $e^o(q)$ maximizes the supplyu chain's revenue net effort cost only if 
 \begin{equation}
     \frac{\partial\Pi(q,e^o(q))}{\partial e}=p\frac{\partial S(q,e^o(q))}{\partial e}-g^\prime(e^o(q))=0.
 \end{equation}
 With a \textbf{buyback contract} the retailer's profit function is
 $$
 \pi_{\mathrm{r}}\left(q, e, w_{\mathrm{b}}, b\right)=(p-b) S(q, e)-\left(w_{\mathrm{b}}-b\right) q-g(e)
 $$
 For all $b>0$ it holds that
 \begin{equation}
    \frac{\partial \pi_{\mathrm{r}}\left(q, e, w_{\mathrm{b}}, b\right)}{\partial e}<\frac{\partial \Pi(q, e)}{\partial e}
 \end{equation}
 Thus, $e^{\mathrm{o}}$ cannot be the retailer's optimal effort level when $b>0$. But $b>0$ is required to coordinate the retailer's order quantity\footnote{\autoref{eq:2.5} and $\lambda\in(0,1)$}, so it follows that the buyback contract cannot coordinate in this setting.

With a \textbf{quantity-flexibility} contract, we have 
$$
\pi_{\mathrm{r}}\left(q, e, w_{\mathrm{q}}, \delta\right)=p S(q, e)-w_{\mathrm{q}}\left(q-\int_{(1-\delta) q}^{q} F(y \mid e) \mathrm{d} y\right)-g(e) .
$$
For all $\delta>0$ (which is required to coordinate the retailer's quantity decision)
$$
\frac{\partial \pi_{\mathrm{r}}\left(q, e, w_{\mathrm{q}}, \delta\right)}{\partial e}<\frac{\partial \Pi(q, e)}{\partial e} .
$$
As a result, the retailer chooses a \textbf{lower effort} than optimal\footnote{Because the left-hand side will first approach $0$, i.e., the retailer will choose effort level lower than $e^o(q)$.}, i.e., the quantity-flexibility contract also does not coordinate the supply chain in this setting.

Also, it can be shown that \textbf{revenue-sharing} contract with $\phi<1$ has
$$\frac{\partial \pi_{\mathrm{r}}\left(q, e, w_{\mathrm{r}}, \phi\right)}{\partial e}<\frac{\partial \Pi(q, e)}{\partial e} .$$ The \textbf{sales-rebate} contract with $r>0$ and $q>t$ has
$$\frac{\partial \pi_{\mathrm{r}}\left(q, e, w_{\mathrm{s}}, r,t\right)}{\partial e}>\frac{\partial \Pi(q, e)}{\partial e},$$
which means the retailer exerts too much effort.

Consider \textbf{quantity discount} contract\footnote{The quantity discount should let the retailer retain the revenues but charge a marginal cost based on expected revenue conditional on the optimal effort.}. 
Suppose $T_d(q)=w_d(q)q$, where 
\begin{equation*}
    w_d(q)=(1-\lambda)p\left(\frac{S(q,e^o)}{q}\right)+\lambda c+(1-\lambda)\frac{g(e^o)}{q})
\end{equation*}
and $\lambda\in[0,1]$. 

Now the retailer's profit function is 
\begin{equation*}
    \pi_r(q,e)=p S(q,e)-(1-\lambda)p S(q,e^o)-\lambda c q-g(e)+(1-\lambda)g(e^o)
\end{equation*}
Given the optimal effort $e^o$, the retailer's profit function is 
$$\pi_r(q,e^o)=\lambda p S(q,e^o)-\lambda cq-\lambda g(e^o)=\lambda\Pi(q,e^o),$$
and so the retailer's optimal order quantity is $q^o$, any allocation of profit is feasible and the supplier's optimal production is $q^o$.

Also, if the demand is dependent on price and effort, let 
\begin{equation*}
    w_d(q)=(1-\lambda)p^o \left(\frac{S(q,e^o)}{q}\right)+\lambda c+(1-\lambda)\frac{g(e^o)}{q}.
\end{equation*}
Again, the retailer retains all revenue and so optimizes price and effort\footnote{The logic.}. Futhermore, the quantity decision is not distorted because the quantity-discount schedule is contingent on the optimal price and effort and not on the chosen price and effort. 




% \nocite{*}
% \printbibliography
\end{document}