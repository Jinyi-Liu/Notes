\documentclass[12pt]{article}
\usepackage[utf8]{inputenc}	
\usepackage{amsmath,amsthm,amsfonts,amssymb,amscd}
\usepackage{multirow,booktabs}
\usepackage[table]{xcolor}
\usepackage{fullpage}
\usepackage{lastpage}
\usepackage{enumitem}
\usepackage{fancyhdr}
\usepackage{mathrsfs}
\usepackage{wrapfig}
\usepackage{setspace}
\usepackage{calc}
\usepackage{multicol}
\usepackage{cancel}
\usepackage[retainorgcmds]{IEEEtrantools}
\usepackage[margin=4cm]{geometry}
\usepackage{amsmath}
\newlength{\tabcont}
\setlength{\parindent}{0.0in}
\setlength{\parskip}{0.00in}
\usepackage{empheq}
\usepackage{framed}
\usepackage[most]{tcolorbox}
\usepackage{xcolor}
\usepackage{hyperref}
\usepackage{palatino}

% \usepackage[style=alphabetic,sorting=nyt,sortcites=true,autopunct=true,babel=hyphen,hyperref=true,abbreviate=false,backref=true,backend=biber]{biblatex}
% \addbibresource{bibliography.bib}

\colorlet{shadecolor}{orange!15}
\parindent 0in
\parskip 8pt
% \geometry{margin=1in, headsep=0.25in}
\theoremstyle{definition}
\newtheorem{defn}{Definition}
\newtheorem{reg}{Rule}
\newtheorem{exer}{Exercise}
\newtheorem{note}{Note}

% \usepackage{amsmath}
\DeclareMathOperator*{\argmax}{arg\,max}
\DeclareMathOperator*{\argmin}{arg\,min}
\newcommand{\oF}{\overline{F}(q)}

\begin{document}


\setcounter{section}{1}
\title{Chapter 2 Review Notes}

\thispagestyle{empty}

\begin{center}
{\LARGE \bf Notes}\\
% {\large Physics 300}\\
\end{center}

% # Supply Chain Coordination
\section{Coordinating the newsvendor}

With the standard wholesale-price contract, it is shown that the retailer does not order enough inventory to maximize the supply chain’s total profit because the retailer ignores the impact of his action on the supplier’s profit. Hence, coordination requires that the retailer be given an **incentive** to increase his order

Types of contracts to coordiante the supply chain and arbitrarily divide its profit:
- Buyback contracts
- revenue-sharing contracts
- quantity-flexibility contracts 
- sales-rebate contracts
- quantity-discount contracts

% ### To do list

\subsection{Model and analysis}
$\mu=E[D]$ is the mean of demand. The supplier's production cost per unit is $c_s$ and the retailer's marginal cost per unit is $c_r$, $c_s+c_r<p$. $c_r$ is incurred upon procuring a unit. Goodwill penalty cost $g_r$ and the analogous cost for the supplier is $g_s$. Let $c=c_s+c_r$ and $g=g_s+g_r$. $\upsilon$ is net of any salvage expenses. 

The details of the negotiation process is not explored.

Each firm is risk neutral. Full information.

- voluntary compliance
- forced compliance
- The approach taken in this section is to assume forced compliance but to check if the supplier has an incentive to deviate from the proposed contractual terms.


\begin{align*}
    S(q)&=E[\min(q,D)]=q(1-F(q))+\int_0^q y f(y)dy=q-\int_0^q F(y)dy\\
    I(q)&=E[(q-D)]^+=q-S(q)\\
    L(q)&=E[(D-q)^+]=\mu-S(q)
\end{align*}

where $I(q)$ is the expected leftover inventory and $L(q)$ is the lost-sales function.

The retailer's profit function is 
\begin{align*}
    \pi_r(q)&=pS(q)+\upsilon I(q)-g_r L(q)-c_r q - T\\
    &=(p-\upsilon+g_r)S(q)-(c_r-\upsilon)q-g_r\mu-T,
\end{align*}

the supplier's profit function is 
\begin{equation*}
    \pi_s(q)=g_s S(q)-c_s q-g_s\mu + T,
\end{equation*}
and the supply chain's profit function is 
\begin{equation}
    \Pi(q)=\pi_r(q)+\pi_s(q)=(p-\upsilon+g)S(q)-(c-\upsilon)q-g\mu
\end{equation}

Let $q^o$ be a supply chain optimal order quantity, we have 
\begin{equation}
    S^\prime(q^o)=\overline F(q^o)=\frac{c-\upsilon}{p-\upsilon+g}
\end{equation}

since $F$ is strictly increasing and thus $\Pi$ is strictly concave and the optimal order quantity is unique.

Let $q_r^*=\argmax \pi_r(q)$

\subsection{The wholesale-price contract}\
Let $T_w(q,w)=w q$. Since $\pi_r(q,w)$ is strictly concave in $q$, we have 

\begin{equation}
    (p-\upsilon+g_r)S^\prime(q_r^*)-(w+c_r-\upsilon)=0.
\end{equation}

Since $S^\prime(q)$ is decreasing, $q_r^*=q^o$ only when
$$
w=(\frac{p-\upsilon+g_r}{p-\upsilon+g})(c-\upsilon)-(c_r-\upsilon).
$$
It shows that $w\leq c_s$, i.e., coordinates only if the supplier earns a \textbf{nonpositive} profit. Thus the wholesale-price contract is generally \textbf{not considered} a coordinating contract.



a







   
\numberwithin{equation}{section}
\newcommand{\spq}{S(q,p^o(q))}
\newcommand{\pq}{p^o(q)}
\thispagestyle{empty}
\section{Coordinating the newsvendor with \textit{price-dependent} demand}


\subsection{Model and analysis}
Now the retailer chooses his price in addition to his order quantity. Let $F(q|p)$ be the distribution function of demand, where $p$ is the retail price. Assume $\frac{\partial F(q|p)}{\partial p}>0$. To obtain initial insights, assume the retailer sets his \textbf{price} \textit{at the same time as} his \textbf{stocking decision} and the price is \textbf{fixed} throughout the season.
\begin{note}
    van Mieghem and Dada (1999). A hybrid model. The retailer chooses $q$, then observes a demand signal and then chooses price.
\end{note}

The integrated channel's profit is 
\begin{equation*}
    \Pi(q,p)=(p-v+g)S(q,p)-(c-v)q-g\mu
\end{equation*}
where $S(q,p)$ is expected sales given the stocking quantity $q$ and the price $p$, and similarly, we have
\begin{equation*}
    S(q,p)=q-\int_0^q F(y|p)dy
\end{equation*}
\begin{note}
    The integrated channel profit function need not be concave nor unimodal (Petruzzi \& Dada 1999)
\end{note}
Let $\pq$ be the supply chain optimal price for a given $q$. The necessary condition for coordination is
\begin{equation}\label{eq:3.1}
    \frac{\partial\Pi(q,p^o(q))}{\partial p}=S(q,p^o(q))+(p^o(q)-v+g)\frac{\partial S(q,p^o(q))}{\partial p}=0.
\end{equation}
\begin{note}
    Either not satisfy the first-order condition or fail to coordinate the quantity decision.
\end{note}

Consider the quantity-flexibility contract. The retailer's profit function is 
\begin{align*}
    \pi_r(q,p,w_q,\delta)=&(p-v+g_r)S(q,p)-(w_q+c_r-v)q\\
    +&(w_q+c_r-v)\int_{(1-\delta)q}^q F(y|p)dy-\mu g_r
\end{align*}
For price coordination the first-order condition must hold,

\begin{align}
\frac{\partial \pi_{\mathrm{r}}\left(q, p^{\mathrm{o}}(q), w_{q}, \delta\right)}{\partial p}=& S\left(q, p^{\mathrm{o}}(q)\right)+\left(p^{\mathrm{o}}(q)-v+g_{\mathrm{r}}\right) \frac{\partial S\left(q, p^{\mathrm{o}}(q)\right)}{\partial p} \nonumber\\
&+\left(w_{q}+c_{\mathrm{r}}-v\right) \int_{(1-\delta) q}^{q} \frac{\partial F\left(y \mid p^{\mathrm{o}}(q)\right)}{\partial p} \mathrm{~d} y \nonumber\\
=& 0\label{eq:3.2}
\end{align}
The second term in \autoref{eq:3.2} is no smaller than the second term in \autoref{eq:3.1}\footnote{The assumption of $\partial F(q|p)/\partial p>0.$}, so the above holds only if the third term is nonpositive. 

\thispagestyle{empty}
\section{Coordinating the newsvendor with effort-dependent demand}

\section{Coordination with multiple newsvendors}
This section considers two models with one supplier and multiple competing retailers. 

\subsection{Competing newsvendors with a fixed retail price}
Set $c_r=g_r=g_s=v=0$, increase the number of retailers to $n>1$. $D$ the total retail demand. And for each retailer $i$'s demand:
\begin{equation*}
    D_i=\left(\frac{q_i}{q}\right)D,
\end{equation*}
where $q=\sum_{i=1}^n q_i$ and $q_{-i}=q-q_i$. 
Given the proportional allocation rule, the integrated supply chain faces a single newsvendor problem. Hence we have 
\begin{equation}
    F(q^o)=\frac{p-c}{p}.
\end{equation}
Retailer $i$'s profit function with a buyback contract is 
\begin{equation*}
    \pi_i(q_i,q_{-i})=(p-w)q_i-(p-b)\left(\frac{q_i}{q}\right)\int_0^q F(x)dx.
\end{equation*}
The above also provides the retailer's profit with a wholesale-price contract (i.e., set $b=0$). It's strictly concave in $q$. Hence, for every $q_{-i}$ there is a unique optimal response. Consider a Nash equilibrium $\{q^*_i\}_{i=1}^n$, it must have
\begin{equation*}
    \frac{\partial\pi_i(q_i,q_{-i})}{\partial q_i}=q^*\left(\frac{p-w}{p-b}\right)-q^*_i F(q^*)-q_{-i}^*\left(\frac{1}{q^*}\int_0^{q^*}F(x)dx\right)=0.
\end{equation*}
Substitute $q_{-i}^{*}=q^{*}-q_{i}^{*}$ into the above equation and solve for $q_{i}^{*}$ given a fixed $q^{*}$ :
\begin{equation}\label{eq:5.2}
    q_{i}^{*}=q^{*} \frac{\left((p-w) /(p-b)-\left(1 / q^{*}\right) \int_{0}^{q^{*}} F(x) \mathrm{d} x\right)}{F\left(q^{*}\right)-1 / q^{*} \int_{0}^{q^{*}} F(x) \mathrm{d} x} .
\end{equation}
Now substitute it into $q^*=n q_i^*$, then we have
\begin{equation}\label{eq:5.3}
    g(q^*)\equiv\frac{1}{n}F(q^*)+\left(\frac{n-1}{n}\right)\left(\frac{1}{q^*}\int_0^{q^*}F(x)dx\right)=\frac{p-w}{p-b}.
\end{equation}
It's easy to see that $g(0)=1$, $g(\infty)=1$ and $g^\prime(\cdot)>0$. Thus, when $b<w<p$, there exists a unique $q^*$ satisfying \autoref{eq:5.3}.

Consider $n$. LHS in \autoref{eq:5.3} is decreasing in $n$, thus $q^*$ is increasing in $n$. Competition makes the retailers order more inventory because of the \textbf{demand-stealing effect}: each retailer \textbf{ignores} the fact that ordering more means the other retailers' demands \textbf{stochastically decrease}.

Due to the \textbf{demand-stealing effect} the supplier can coordinate the supply chain and earn a positive profit with just a wholesale-price contract.
Let $\hat{w}(q)$ be the wholesale price that induces the retailers to order $q$ units with a wholesale-price contract (i.e., with $b=0$). From \autoref{eq:5.3},
\begin{equation*}
    \hat{w}(q)=p\left(1-\left(\frac{1}{n}\right) F(q)-\left(\frac{n-1}{n}\right)\left(\frac{1}{q} \int_{0}^{q} F(x) \mathrm{d} x\right)\right).
\end{equation*}
By definition $\hat{w}(q^o)$ is the coordinating wholesale price. Given $F(q^o)=(p-c)/c$ and 
\begin{equation*}
    \frac{1}{q}\int_0^q F(x)dx<F(q),
\end{equation*}
it can be shown that $\hat{w}(q^o)>c$ when $n>1$\footnote{
    $\hat{w}(q^o)>p\left(1-F(q^o)\right)=p(1-\frac{p-c}{c})=\frac{2cp-p^2}{c}$ ?????????\textbf{TBD.}
}. Hence, the supplier earns a positive profit. But with the \textbf{single} retailer model channel coordination is only achieved when the supplier earns zero profit, i.e., $\hat{w}(q^o)=c$. 

But the coordination is not optimal for supplier. The profit function is
\begin{equation*}
    \pi_s(q,\hat{w}(q))=q(\hat{w}(q)-c).
\end{equation*}
Assuming $n>1$, we have 
\begin{equation*}
    \frac{\partial\pi_s(q^o,\hat{w}(q^o))}{\partial q}=-\frac{q^o p f(q^o)}{n}<0.
\end{equation*}








\section{Coordinating the newsvendor with demand updating}

\newpage
\section{Coordination in the single-location base-stock model}
This section considers a model with perpetual demand and many replenishment opportunities. Hence, the newsvendor model is not appropriate. 

\subsection{Model and analysis}
Suppose a supplier sells a single product to a single retailer. Let $L_r$ be the lead time to replenish an order from the retailer. The supplier has infinite capacity so the supplier keeps no inventory and the retailer's replenishment lead time is always $L_r$. Let $\mu_r=E[D_r]$ and $F_r,f_r$ of $D_r$ with $F_r$ strictly increasing and $F_r(0)=0$, which rules out the possibility that it is optimal to carry no inventory. 

The retailer incurs inventory holding costs at rate $h_r>0$ per unit of inventory. $\beta_r,\beta_s$ the backorder penalty. 

Let $I_r(y)$ be the retailer's expected inventory at time $t+L_r$ when the retailer's inventory level is $y$ at time $t$:
\begin{equation}
    I_r(y)=\int_0^y (y-x)f_r(x)dx=\int_0^y F_r(x)dx.
\end{equation}
Let $B_r(y)$ be the expected backorders
\begin{equation}
    B_r(y)=\int_y^\infty (x-y)f_r(x)dx=\mu_r-y+I_r(y).
\end{equation}
With a base-stock policy, the retailer continuously orders inventory and chooses $s_r$.

Let $c_r(s_r)$ be the retailer's average cost per unit time when the retailer implements the base-stock policy $s_r$:
\begin{align*}
    c_r(s_r)&=h_r I_r(s_r)+\beta_r B_r(s_r)\\
    &=\beta_r(\mu_r-s_r)+(h_r+\beta_r)I_r(s_r).
\end{align*}
The supplier's expected cost function is 
\begin{align*}
    c_s(s_r)&=\beta_s B_r(s_r)\\
    &=\beta_s(\mu_r-s_r+I_r(s_r)).
\end{align*}
Let $c(s_r)$ be the supply chain's expected cost per unit time,
\begin{align}
    c(s_r)&=c_r(s_r)+c_s(s_r)\nonumber\\
    &=\beta(\mu_r-s_r)+(h_r+\beta)I_r(s_r).\label{eq:7.3}
\end{align}
$c\left(s_{\mathrm{r}}\right)$ is strictly convex, so there is a unique supply chain optimal base-stock level, $s_{\mathrm{r}}^{\mathrm{o}}$. It satisfies the following critical ratio equation
$$
I_{\mathrm{r}}^{\prime}\left(s_{\mathrm{r}}^{\mathrm{o}}\right)=F_{\mathrm{r}}\left(s_{\mathrm{r}}^{\mathrm{o}}\right)=\frac{\beta}{h_{\mathrm{r}}+\beta}
$$
Let $s_{\mathrm{r}}^{*}$ be the retailer's optimal base-stock level. The retailer's cost function is also strictly convex, so $s_{\mathrm{r}}^{*}$ satisfies
$$
F_{\mathrm{r}}\left(s_{\mathrm{r}}^{*}\right)=\frac{\beta_{\mathrm{r}}}{h_{\mathrm{r}}+\beta_{\mathrm{r}}} .
$$
Given 	$\beta_r<\beta$, it follows from the above two expressions that $s_r^* < s_r^o$, i.e., the retailer chooses a base-stock level that is \textbf{less than optimal}.

Suppose the supplier agree to transfer at every time $t$
$$t_I I_r(y)+t_B B_r(y)$$
where $y$ is the retailer's inventory level at time $t$ and $t_I$ and $t_I$ are constants. Further more, consider $\lambda\in(0,1]$\footnote{If $\lambda=0$, then any base-stock level is optimal},
\begin{align*}
    t_I&=(1-\lambda)h_r\\
    t_B&=\beta_r-\lambda\beta
\end{align*}
The retailer's expected cost function is now 
\begin{equation}\label{eq:7.5}
    c_r(s_r)=(\beta_r-t_B)(\mu_r-s_r)+(h_r+\beta_r-t_I-t_B)I_r(s_r).
\end{equation}
It follows from \autoref{eq:7.3} and \autoref{eq:7.5} that
\begin{equation}
    c_r(s_r)=\lambda c(s_r)\label{eq:7.6}
\end{equation}
Hence, $s_r^o$ minimizes the retailer's cost and the contracts coordinate the supply chain. 





\newpage
\section{Coordination in the two-location base-stock model}
\subsection{Model}
Let $h_s$, $0<h_s<h_r$, be the supplier's per unit holding cost rate incurred with on-hand inventory. Let $D_s<0$ be demand during an interval of time with length $L_s$. 

Both firms use base-stock policies to manage inventory.

\subsection{Cost function}
Let $c_i(s_r,s_s)$ be the average rate at which firm $i$ incurs costs at the retail level. 
At time $t+L_s$, the retailer's inventory level is $s_r-\left(D_s-s_s\right)^+$. So
\begin{equation*}
    c_i(s_r,s_s)=F_s(s_s)c_i(s_r)+\int_{s_s}^\infty c_i(s_r+s_s-x)f_s(x)dx.
\end{equation*}
Let $I_r(s_r,s_s)$ and $B_r(s_r,s_s)$ be the retailer's average inventory and backorders given the base-stock levels:
\begin{align*}
    I_r(s_r,s_s)&=F_s(s_s)I_r(s_r)+\int_{s_s}^\infty I_r(s_r+s_s-x)f_s(x)dx\\
    B_r(s_r,s_s)&=F_s(s_s)B_r(s_r)+\int_{s_s}^\infty B_r(s_r+s_s-x)f_s(x)dx\\
\end{align*}
Let $\pi_i(s_r,s_s)$ be firm $i$'s total average cost rate. We have 
$$\pi_r(s_r,s_s)=c_r(s_r,s_s).$$
Let $I_s(s_s)$ be the supplier's average inventory. Analogous to the retailer's functions, we have 
$$I_s(y)=\int_0^y F_s(x)dx.$$
The supplier's average cost is 
$$\pi_s(s_r,s_s)=h_s I_s(s_s)+c_s(s_r,s_s).$$
The supply chain's total cost is $\Pi(s_r,s_s)=\pi_r(s_r,s_s)+\pi_s(s_r,s_s)$.

\subsection{Behavior in the decentralized game}
Let $s_i(s_j)$ be an optimal base-stock level for firm $i$ given firm $j$'s strategy. 
A Nash equilibrium $\{s_r^*,s_s^*\}$ is $s_r^*=s_r (s_s^*)$ and $s_s^*=s_s(s_r^*)$. It's unique and exists. But the penalty is positive. The uniqueness requires
\begin{equation}
    |s_i^\prime(s_j)|<1.
\end{equation}

\subsection{Coordiantion with linear transfer payments}
Supply chain coordination in this setting is achieved when $\{s_r^o,s_s^o\}$ is a Nash equilibrium. 

Suppose the supplier offers
\begin{equation*}
    t_I I_r(s_r,s_s)+t_B^r B_r(s_r,s_s)+t_B^s B_s(s_s),
\end{equation*}
where $t_I, t_B^r$ and $t_B^s$ are constants and $B_s(s_s)$ is the supplier's average backorder:
\begin{equation*}
    B_s(y)=\mu_s-y+I_s(y).
\end{equation*}

Given $\Pi\left(s_{\mathrm{r}}, s_{\mathrm{s}}\right)$ is continuous, any optimal policy with $s_{\mathrm{s}}>0$ must set the following two marginals to zero
\begin{equation}
    \frac{\partial \Pi\left(s_{\mathrm{r}}, s_{\mathrm{s}}\right)}{\partial s_{\mathrm{r}}}=F_{\mathrm{s}}\left(s_{\mathrm{s}}\right) c^{\prime}\left(s_{\mathrm{r}}\right)+\int_{s_{\mathrm{s}}}^{\infty} c^{\prime}\left(s_{\mathrm{r}}+s_{\mathrm{s}}-x\right) f_{\mathrm{s}}(x) \mathrm{d} x
\end{equation}
and
\begin{equation}
    \frac{\partial \Pi\left(s_{\mathrm{r}}, s_{\mathrm{s}}\right)}{\partial s_{\mathrm{s}}}=F_{\mathrm{s}}\left(s_{\mathrm{s}}\right) h_{\mathrm{s}}+\int_{s_{\mathrm{s}}}^{\infty} c^{\prime}\left(s_{\mathrm{r}}+s_{\mathrm{s}}-x\right) f_{\mathrm{s}}(x) \mathrm{d} x .
\end{equation}
Since $F_{\mathrm{s}}\left(s_{\mathrm{s}}\right)>0$, there is only one possible optimal policy with $s_{\mathrm{s}}>0,\left\{\tilde{s}_{\mathrm{r}}^{1}, \tilde{\mathrm{s}}_{\mathrm{s}}^{1}\right\}$, where $\tilde{s}_{\mathrm{r}}^{1}$ satisfies
\begin{equation}\label{eq:8.4}
    c^{\prime}\left(\tilde{s}_{\mathrm{r}}^{1}\right)=h_{\mathrm{s}},
\end{equation}
and $\tilde{s}_s^1$ satisfies $\partial\Pi(\tilde{s}_r^1,\tilde{s}_s^1)/\partial s_s=0$. \autoref{eq:8.4} simplifies 
\begin{equation*}
    F_r(\tilde{s}_r^1)=\frac{h_s+\beta}{h_r+\beta},
\end{equation*}
so $\tilde{s}_r^1$ exists and is unique. Also, $\Pi(\tilde{s}_r^1,s_s)$ is strictly convex in $s_s$. $\tilde{s}_s^1$ also exists and is unique.

\begin{note}
    There are too many questions. TBD.
\end{note}








\newpage
\section{Coordination with internal markets}
\subsection{Model and analysis}
Suppose there is one supplier, one production manager and two retailers. The production manager is the supplier's employee and the retailers are independent firms.

\begin{itemize}
    \item The production manager chooses a product input level $e$, which yields an output of $Q=Y e$ finished units, where $Y\in\left[0,1\right]$ is a random variable.
    \item The production manager incurs cost $c(e)$, where $c(e)$ is strictly convex and increasing.
    \item Retailer $i$ observes $\alpha_i$, the realization of $A_i>0$. Each retailer submits an order to the supplier.
    \item The supplier allocates $q_i$ to retailer $i$ with $q_i$ not exceeding the order and $q_1+q_2\leq Q$.
    \item Retailer $i
    $ earns $q_i p_i(q_i)$ where $p_i(q_i)=\alpha_i q_i^{-1/\eta}$ and $\eta>1$ is the constant demand elasticity. 
    \item Let $\theta$ be the realization of $Y$, $A=\left\{A_1,A_2\right\} $and $\alpha=\left\{\alpha_1,\alpha_2\right\}$.
\end{itemize}

\textbf{First consider the supply chain optimal actions}. 
Let $\gamma$ be the fraction of $Q$ that is allocated to retailer $1$. The total retailer revenue is 
\begin{equation*}
    \pi(\gamma,\alpha,Q)=\left(\alpha_1 \gamma^{(\eta-1)/\eta}+\alpha_2 (1-\gamma)^{(\eta-1)/\eta}\right)Q^{(\eta-1)/\eta}.
\end{equation*}
Let $\gamma^{\circ}(\alpha)$ be the optimal share to allocate to retailer one:
\begin{equation}
    \gamma^{o}(\alpha)=\frac{\alpha_{1}^{\eta}}{\alpha_{1}^{\eta}+\alpha_{2}^{\eta}}
\end{equation}
Conditional on an optimal allocation, the retailers' total revenue is
$$
\pi(\alpha, Q)=\pi\left(\gamma^{\circ}(\alpha), \alpha, Q\right)=\left(\alpha_{1}^{\eta}+\alpha_{2}^{\eta}\right)^{1 / \eta} Q^{(\eta-1) / \eta} .
$$
Total expected supply chain profit, $\Pi(e, A, Y)$, is thus
$$
\Pi(e, A, Y)=E[\pi(A, Y e)]-c(e) .
$$
Profit is concave in $e$, so the unique optimal production effort level, $e^{\mathrm{o}}$, satisfies
\begin{equation}\label{eq:9.2}
    \Pi^{\prime}\left(e^{\mathrm{o}}\right)=\left(\frac{\eta-1}{\eta}\right)\left(e^{\mathrm{o}}\right)^{-1 / \eta} E\left[\left(A_{1}^{\eta}+A_{2}^{\eta}\right)^{1 / \eta} Y^{(\eta-1) / \eta}\right]-c^{\prime}\left(e^{\mathrm{o}}\right)=0.
\end{equation}

Now consider \textbf{decentralized operations}. 
Suppose the supplier charges the reatilers $w$ per unit. Then retailer $i$'s profit is 
\begin{equation*}
    \pi_i(q_i,w)=\alpha_i q_i^{(\eta-1)\eta}-w q_i.
\end{equation*}
Retailer $i$'s optimal quantity, $q_{i}^{*}$, satisfies
$$
\frac{\partial \pi_{i}\left(q_{i}^{*}, w\right)}{\partial q_{i}}=0=\left(\frac{\eta-1}{\eta}\right) \alpha_{i}\left(q_{i}^{*}\right)^{-1 / \eta}-w .
$$
It follows that $q_{i}^{*}=\gamma^{\circ}(\alpha) Q$ when $w=w(\alpha, Q)$,
$$
w(\alpha, Q)=\left(\frac{\eta-1}{\eta}\right)\left(\alpha_{1}^{\eta}+\alpha_{2}^{\eta}\right)^{1 / \eta} Q^{-1 / \eta}
$$
\begin{note}
    $$\frac{\partial\pi(\alpha,Q)}{\partial Q}=w(\alpha,Q).$$
\end{note}

The supplier could offer the retailers a contract that identifies $w(\alpha,Q)$ as the wholesale price contingent on the realization of $A$ and $Y$. The supplier merely commits at the start of the game to hold a market for output after the retailers observe their demand realizations. The unique market-clearing price is $w(\alpha,Q)$, and so the market optimizes the supply chain's profit without the supplier observing $A$.

Consider the\textbf{ production manager}. Assume the supplier can't observe $e$. But she can observe the final output $Ye$. So suppose the supplier pays the production manager
\begin{equation}\label{eq:9.3}
    \begin{aligned}
        &\left(\frac{\eta-1}{\eta}\right)\left(e^{0}\right)^{-1 / \eta} E\left[\left(A_{1}^{\eta}+A_{2}^{\eta}\right)^{1 / \eta} Y^{(\eta-1) / \eta}\right] / E[Y]\\
        =&E\left[Q w(A, Q) \mid e^{0}\right] / E\left[Q \mid e^{\mathrm{o}}\right]
    \end{aligned}
\end{equation}
per unit of realized output. 
Hence, the supplier pays the production manager the \textbf{expected shadow price} of capacity and sells capacity to the retailer for the \textbf{realized shadow price} of capacity. The production manager's expected utility is 
$$
u(e)=\left(\frac{\eta-1}{\eta}\right)\left(e^{\mathrm{o}}\right)^{-1 / \eta} E\left[\left(A_{1}^{\eta}+A_{2}^{\eta}\right)^{1 / \eta} Y^{(\eta-1) / \eta}\right] e-c(e)
$$
and the marginal utility is
$$
u^{\prime}(e)=\left(\frac{\eta-1}{\eta}\right)\left(e^{\mathrm{o}}\right)^{-1 / \eta} E\left[\left(A_{1}^{\eta}+A_{2}^{\eta}\right)^{1 / \eta} Y^{(\eta-1) / \eta}\right]-c^{\prime}(e) .
$$
From \autoref{eq:9.2} we know the production manager's optimal effort is $e^o$. 
\begin{note}
    The supplier earns zero profit in expectation from the internal market. $E[Qw(A,Q)|e^o]$ is the expected revenue from the retailers, and from \autoref{eq:9.3}, it's also the expected payout to the production manager.
    Kouvelis and Lariviere (2000) 
\end{note}






\newpage
\section{Asymmetric information}
Based on Cachon and Lariviere (2001). In this model coordination requires (1) the supplier takes the correct action and (2) an accurate demand forecast is shared.

\subsection{The capacity procurement game}
In the capacity procurement game a manufacturer, $M$, develops a new product with uncertain demand. A single potential supplier, $S$, produces a critical component. Let $D_\theta$ be demand, where $\theta\in\{h,l\}$. Let $F(x\mid\theta)$ be the distribution function of demand and $D_h$ stochastically dominates $D_l$. 

With asymmetric information the $\theta$ parameter is observed \textbf{only by the manufacturer}. $P(\theta=h)=\rho$ where $\rho$ is a commom knowledge.

The interactions are divided into two states. 
\begin{enumerate}
    \item $M$ gives $S$ a demand forecast and offers $S$ a contract which includes an inital order, $q_i$. Assuming $S$ accepts the contract and constructs $k$ units of capacity at a cost $c_k>0$ per unit. 
    \item $M$ observes $D_\theta$ and places her final order with $S$, $q_f$, where the contract specifies the set of feasible final orders. Then $S$ produces $\min\{D_\theta,k\}$ units at a cost of $c_p>0$ per unit. $M$ pays $S$ based on the agreed contract and $M$ earns $r>c_p+c_k$ per unit of demand satisfied. The salvage value of unused units of capacity is normalized to zero. 
\end{enumerate}

\subsection{Full information}
Let $S_\theta(x)$ be expected sales with $x$ units of capacity,
\begin{align*}
    S_\theta(x)&=x-E\left[\left(x-D_\theta\right)^+\right]\\
    &=x-\int_0^x F_\theta(x)dx.
\end{align*}
Let $\Omega_\theta(k)$ be the supply chain's expected profit with $k$ units of capacity,
\begin{equation*}
    \Omega_\theta(k)=(r-c_p)S_\theta(k)-c_k k.
\end{equation*}
Given that $\Omega_{\theta}(k)$ is concave, the optimal capacity, $k_{\theta}^{\mathrm{o}}$, satisfies the newsvendor critical ratio:
$$
\overline{F}_{\theta}\left(k_{\theta}^{\mathrm{o}}\right)=\frac{c_{k}}{r-c_{p}},
$$
Let $\Omega_\theta^o=\Omega_\theta(k_\theta^o)$. The supply chain can be coordinated with building $K_\theta^o$ units of capacity.

Now consider the game between $M$ and $S$. Consider an options contract: $M$ purchases $q_i$ options for $w_o$ per option at state 1 and then pays $w_e$ to exercise each option at stage $2$. Hence, the expected transfer payment is 
\begin{equation*}
    w_o q_i+w_e S_\theta(q_i).
\end{equation*}

Assuming $k=q_i$, the manufacturer's expected profit is
\begin{equation*}
    \Pi_\theta(q_i)=(r-w_e)S_\theta(q_i)-w_o q_i.
\end{equation*}
Then choose parameters such that $r-w_e=\lambda(r-c_p)$ and $w_o=\lambda c_k$ where $\lambda\in[0,1]$, we have
\begin{equation*}
    \Pi_\theta(q_i)=\lambda\Omega_\theta(q_i).
\end{equation*}
Hence, $q_i=k_\theta^o$ is the manufacturer's optimal order. It maximizes the supplier's profit, apparently confirming the initial $k=q_i$ assumption.

However, the manufacturer \textbf{can't} be sure the supplier indeed builds $k=q_i$. Consider the following profit function for a supplier (assuming $k<q_i$) who \textbf{believe} demand is $\tau$,
\begin{align*}
    \pi(k,q_i,\tau)&=(w_e-c_p)s_\tau(k)+w_o q_i - c_k k\\
    &=(1-\lambda)(r-c_p)s_\tau(k)-c_k(k-\lambda q_i).
\end{align*}
Then we find that
$$\frac{\partial\pi(k_\theta^o,k_\theta^o,\theta)}{\partial k}<0,$$
i.e., $k_\theta^o$ does not maximizes the supplier's profit if $q_i=k_\theta^o$. 
\begin{note}
    $w_o$ has nothing do with the contract term $k$ in the profit function, so the supplier sets his capacity as if the supplier is offered just a wholesale-price contract $w_e$. 
\end{note}

To influence the supplier's capacity decision the manufacturer is relegated to a contract \textbf{based on his final order}, $q_f$. An obvious candidate is the wholesafe-price contract. 











% \nocite{*}
% \printbibliography
\end{document}